\documentclass[a4paper,12pt]{article}

\usepackage{geometry}
\geometry{a4paper,left=18mm,right=18mm, top=2cm, bottom=2cm} 

\usepackage{lmodern}
\usepackage[T1]{fontenc}
\usepackage[utf8]{inputenc}
\usepackage[ngerman]{babel}
\usepackage[solution_on, random=0]{srdp-mathematik} % solution_on/off, random

\pagestyle{plain} %PAGESTYLE: empty, plain
\onehalfspacing %Zeilenabstand
\setcounter{secnumdepth}{-1} % keine Nummerierung der Überschriften
\usepackage{blindtext}
%
%
%%%%%%%%%%%%%%%%%%%%%%%%%%%%%%%%%%%%%%%%%%%%%%%%%%%%%%%%%%%%%%%%%
%%%%%%%%%%%%%%%%%%%%%% DOKUMENT - ANFANG %%%%%%%%%%%%%%%%%%%%%%%%
%%%%%%%%%%%%%%%%%%%%%%%%%%%%%%%%%%%%%%%%%%%%%%%%%%%%%%%%%%%%%%%%%
%
%



\begin{document}
\begin{beispiel}[1/2]{1} %PUNKTE DES BEISPIELS
dasdlk
\end{beispiel}


\begin{langesbeispiel} \item[0] %PUNKTE DES BEISPIELS

\begin{aufgabenstellung}
\item  Lea trinkt eine Tasse Kaffee. In der nachstehenden Abbildung ist der Graph der Funktion K dargestellt, die modellhaft die Konzentration K(t) von Koffein in Leas Blut in Abhängigkeit von
der Zeit t nach dem Trinken des Kaffees beschreibt (t in h, K(t) in mg/L).

\Subitem{Ermitteln Sie mithilfe der obigen Abbildung, wie viele Minuten nach dem Trinken des Kaffees
die maximale Konzentration von Koffein im Blut auftritt.} \pkte[1/2]{1}


\end{aufgabenstellung}



\end{langesbeispiel}

\end{document}
