

\documentclass[a4paper,12pt]{article}
\usepackage{geometry}
\geometry{a4paper,left=18mm,right=18mm, top=2cm, bottom=2cm}
 

\usepackage{lmodern}
\usepackage[T1]{fontenc}
\usepackage[utf8]{inputenc}
\usepackage[ngerman]{babel}
\usepackage[solution_on, random=0, info_off]{srdp-mathematik} % solution_on/off, random, info_on/off
%
\usepackage{blindtext}
%
%\pagestyle{plain} %PAGESTYLE: empty, plain
%\onehalfspacing %Zeilenabstand
%\setcounter{secnumdepth}{-1} % keine Nummerierung der Überschriften
%
%
%%%%%%%%%%%%%%%%%%%%%%%%%%%%%%%%%%%%%%%%%%%%%%%%%%%%%%%%%%%%%%%%%
%%%%%%%%%%%%%%%%%%%%%% DOKUMENT - ANFANG %%%%%%%%%%%%%%%%%%%%%%%%
%%%%%%%%%%%%%%%%%%%%%%%%%%%%%%%%%%%%%%%%%%%%%%%%%%%%%%%%%%%%%%%%%
%
%

%\NewEnviron{test}{
%	\begin{enumerate}
%		\item {\BODY}
%	\end{enumerate}
%}

\begin{document}
%\begin{test}
%Written text
%
%\begin{center}
%dasd
%\end{center}
%
%\end{test}

\begin{beispiel}[1/2][2] %PUNKTE DES BEISPIELS
Aufgabentext

\lueckentext{
				text={das}, 	%Lueckentext Luecke=\gap
				L1={•}, 		%1.Moeglichkeit links  
				L2={•}, 		%2.Moeglichkeit links
				L3={•}, 		%3.Moeglichkeit links
				R1={•}, 		%1.Moeglichkeit rechts 
				R2={•}, 		%2.Moeglichkeit rechts
				R3={•}, 		%3.Moeglichkeit rechts
				%% LOESUNG: %%
				A1=0,   % Antwort links
				A2=0		% Antwort rechts 
				}
\end{beispiel}


\begin{beispiel}{48} %PUNKTE DES BEISPIELS
dsa
\end{beispiel}

\defgesamtpunkte{24}
\notenschluessel{0.91}{0.8}{0.64}{0.5}
\end{document}

% Aufgabenliste: 4, 10