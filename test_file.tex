

\documentclass[a4paper,12pt]{article}
\usepackage{geometry}
\geometry{a4paper,left=18mm,right=18mm, top=2cm, bottom=2cm}
 

\usepackage{lmodern}
\usepackage[T1]{fontenc}
\usepackage[utf8]{inputenc}
\usepackage[ngerman]{babel}
\usepackage[solution_on, random=0, info_off]{srdp-mathematik} % solution_on/off, random, info_on/off


\pagestyle{plain} %PAGESTYLE: empty, plain
\onehalfspacing %Zeilenabstand
\setcounter{secnumdepth}{-1} % keine Nummerierung der Überschriften
%
%
%%%%%%%%%%%%%%%%%%%%%%%%%%%%%%%%%%%%%%%%%%%%%%%%%%%%%%%%%%%%%%%%%
%%%%%%%%%%%%%%%%%%%%%% DOKUMENT - ANFANG %%%%%%%%%%%%%%%%%%%%%%%%
%%%%%%%%%%%%%%%%%%%%%%%%%%%%%%%%%%%%%%%%%%%%%%%%%%%%%%%%%%%%%%%%%
%
%

\begin{document}
\begin{titlepage}

\flushright
~\vfil 
\textsc{\Huge 1. Mathematikschularbeit}\\ [2cm] 

\textsc{\Large am Donnerstag, 04. November 2021}\\ [1cm] 

\textsc{\Large Klasse } \\ [1cm] 

\Large Name: \rule{8cm}{0.4pt} \\ 

\vfil\vfil\vfil 
\end{titlepage}

\subsubsection{Typ 1 Aufgaben}

\newpage 

\subsubsection{Typ 2 Aufgaben}

\begin{langesbeispiel} \item[8]	% Aufgabe: 4
Manche einjährige Nutz- und Zierpflanzen wachsen in den ersten Wochen nach der Pflanzung sehr rasch. Im Folgenden wird nun eine spezielle Sorte betrachtet. Die endgültige Größe einer
Pflanze der betrachteten Sorte hängt auch von ihrem Standort ab und kann im Allgemeinen zwischen 1,0\,m und 3,5\,m liegen. Pflanzen dieser Sorte, die im Innenbereich gezüchtet werden,
erreichen Größen von 1,0\,m bis 1,8\,m. 

In einem Experiment wurde der Wachstumsverlauf dieser Pflanze im Innenbereich über einen Zeitraum von 17 Wochen beobachtet und ihre Höhe dokumentiert. Im Anschluss wurde die
Höhe $h$ dieser Pflanze in Abhängigkeit von der Zeit t durch eine Funktion $h$ mit $h(t) = \frac{1}{24}\cdot \left(-t^3+27t^2+120\right)$ modelliert. Dabei bezeichnet $t$ die Anzahl der Wochen seit der Pflanzung und $h(t)$ die Höhe zum Zeitpunkt $t$ in Zentimetern. Die nachstehende Abbildung zeigt den Graphen der Funktion h im Beobachtungszeitraum $[0; 17]$.

\begin{center}
\resizebox{0.8\linewidth}{!}{\psset{xunit=0.5cm,yunit=0.05cm,algebraic=true,dimen=middle,dotstyle=o,dotsize=5pt 0,linewidth=0.8pt,arrowsize=3pt 2,arrowinset=0.25}
\begin{pspicture*}(-1.8,-16.388636363636373)(17.764967532467544,130.55535714285685)
\multips(0,0)(0,10.0){13}{\psline[linestyle=dashed,linecap=1,dash=1.5pt 1.5pt,linewidth=0.4pt,linecolor=gray]{c-c}(0,0)(17.764967532467544,0)}
\multips(0,0)(1.0,0){19}{\psline[linestyle=dashed,linecap=1,dash=1.5pt 1.5pt,linewidth=0.4pt,linecolor=gray]{c-c}(0,0)(0,130.55535714285685)}
\begin{scriptsize}
\psaxes[xAxis=true,yAxis=true,Dx=1.,Dy=20.,ticksize=-2pt 0,ysubticks=2, subticksize=1, subtickcolor=black]{->}(0,0)(0,0)(17.764967532467544,130.55535714285685)[$t$,140] [$h(t)$,-40]
\psplot[linewidth=1.2pt,plotpoints=200]{0}{17}{1.0/24.0*(-x^(3.0)+27.0*x^(2.0)+120.0)}

\rput[bl](8.5,70){$h$}
\end{scriptsize}
\end{pspicture*}}
\end{center}%Aufgabentext

\begin{aufgabenstellung}
\item %Aufgabentext

\Subitem{Berechne den Wert des Quotienten $\frac{h(13)-h(9)}{4}$ und den Wert von $h'(9)$.} %Unterpunkt1
\Subitem{Gib an, welche Bedeutung die beiden berechneten Ergebnisse im gegebenen Kontext haben.} %Unterpunkt2

\item %Aufgabentext

\Subitem{Zeige durch Rechnung, dass die Funktion $h$ im gegebenen Intervall keinen lokalen Hochpunkt hat.}
\Subitem{Begründe deine Rechenschritte.}

\item Für das Wachstum der beobachteten Pflanze ist auch die entsprechende Düngung von Bedeutung. Im gegebenen Fall wurde die Pflanze zwei Wochen vor dem Zeitpunkt des stärksten Wachstums gedüngt.%Aufgabentext

\Subitem{Ermittle diesen Zeitpunkt durch Rechnung.} %Unterpunkt1
\Subitem{Begründe deine Überlegungen.} %Unterpunkt2

\item Im selben Zeitraum wurde das Höhenwachstum von zwei weiteren Pflanzen der gleichen Sorte beobachtet und modelliert. Die nachstehenden Abbildungen zeigen die Graphen der
entsprechenden Funktionen $h_1$ und $h_2$.

\meinlr{
\begin{center}
\psset{xunit=0.37cm,yunit=0.03cm,algebraic=true,dimen=middle,dotstyle=o,dotsize=5pt 0,linewidth=0.8pt,arrowsize=3pt 2,arrowinset=0.25}
\begin{pspicture*}(-1.8,-16.388636363636373)(17.764967532467544,130.55535714285685)
\multips(0,0)(0,10.0){15}{\psline[linestyle=dashed,linecap=1,dash=1.5pt 1.5pt,linewidth=0.4pt,linecolor=black!60]{c-c}(0,0)(17.764967532467544,0)}
\multips(0,0)(1.0,0){19}{\psline[linestyle=dashed,linecap=1,dash=1.5pt 1.5pt,linewidth=0.4pt,linecolor=black!60]{c-c}(0,0)(0,130.55535714285685)}
\psaxes[labelFontSize=\scriptstyle,xAxis=true,yAxis=true,Dx=1.,Dy=20.,ticksize=-2pt 0,ysubticks=2, subticksize=1, subtickcolor=black]{->}(0,0)(0,0)(17.764967532467544,130.55535714285685)[t,140] [$h_1(t)$,-40]
\psplot[linewidth=1.2pt,plotpoints=200]{0}{17}{-7.42*2.718281828459045^(-0.2*(x-14.1))+130.0}
\begin{scriptsize}
\rput[bl](3.5,85){$h_1$}
\end{scriptsize}
\end{pspicture*}
\end{center}}
{\begin{center}
\psset{xunit=0.37cm,yunit=0.03cm,algebraic=true,dimen=middle,dotstyle=o,dotsize=5pt 0,linewidth=0.8pt,arrowsize=3pt 2,arrowinset=0.25}
\begin{pspicture*}(-1.8,-16.388636363636373)(17.764967532467544,130.55535714285685)
\multips(0,0)(0,10.0){15}{\psline[linestyle=dashed,linecap=1,dash=1.5pt 1.5pt,linewidth=0.4pt,linecolor=black!60]{c-c}(0,0)(17.764967532467544,0)}
\multips(0,0)(1.0,0){19}{\psline[linestyle=dashed,linecap=1,dash=1.5pt 1.5pt,linewidth=0.4pt,linecolor=black!60]{c-c}(0,0)(0,130.55535714285685)}
\psaxes[labelFontSize=\scriptstyle,xAxis=true,yAxis=true,Dx=1.,Dy=20.,ticksize=-2pt 0,ysubticks=2, subticksize=1, subtickcolor=black]{->}(0,0)(0,0)(17.764967532467544,130.55535714285685)[t,140] [$h_2(t)$,-40]
\psplot[linewidth=1.2pt,plotpoints=200]{0}{17}{5.55*1.21^(x-0.6)}
\begin{scriptsize}
\rput[bl](5.3,20){$h_2$}
\end{scriptsize}
\end{pspicture*}
\end{center}
}%Aufgabentext

\Subitem{Vergleiche das Krümmungsverhalten der Funktionen $h$, $h_1$ und $h_2$ im Intervall $[0; 17]$.} %Unterpunkt1
\Subitem{Interpretiere das Krümmungsverhalten im Hinblick auf das Wachstum der drei Pflanzen.} %Unterpunkt2

\end{aufgabenstellung}

\begin{loesung}
\item \subsection{Lösungserwartung:} 

\Subitem{$\frac{h(13)-h(9)}{4}\approx 9,47$

$h'(t)=\frac{1}{24}\cdot(-3t^2+54t)=\frac{1}{8}\cdot(-t^2+18t) \Rightarrow h'(9)\approx 10,13$} %Lösung von Unterpunkt1
\Subitem{Die mittlere Wachstumsgeschwindigkeit im Zeitintervall $[9;13]$ beträgt rund $9,5\,cm$ pro Woche. Die momentane Wachstumsgeschwindigkeit zum Zeitpunkt $t=9$, d.h. nach 9 Wochen beträgt rund $10,1\,cm$ pro Woche.} %%Lösung von Unterpunkt2

\setcounter{subitemcounter}{0}
\subsection{Lösungsschlüssel:}
 
\Subitem{Ein Punkt für die korrekte Berechnung des Quotienten sowie des Werts $h'(9)$.} %Lösungschlüssel von Unterpunkt1
\Subitem{Ein Punkt für die korrekte Interpretation im Kontext.} %Lösungschlüssel von Unterpunkt2

\item \subsection{Lösungserwartung:} 

\Subitem{$h'(t)=\frac{1}{24}\cdot(-3t^2+54t)=\frac{1}{8}\cdot(-t^2+18t)$
	
	$t\cdot(-t+18)=0$
	
	$t_1=0, t_2=18$} %Lösung von Unterpunkt1
\Subitem{In einem lokalen Hochpunkt muss die Tangente an den Graphen horizontal sein, d.h., die 1. Ableitung muss den Wert 0 haben.
	
	Die Funktion hat an der Stelle $t=0$ ein lokales Minimum und an der Stelle $t=18$ ein lokales Maximum. Der Wert $t=18$ liegt nicht im Beobachtungsintervall, d.h., die Funktion hat im gegebenen Intervall keinen lokalen Hochpunkt.} %%Lösung von Unterpunkt2

\setcounter{subitemcounter}{0}
\subsection{Lösungsschlüssel:}
 
\Subitem{Ein Punkt für die korrekte Berechnung.} %Lösungschlüssel von Unterpunkt1
\Subitem{Ein Punkt für eine korrekte Begründung.} %Lösungschlüssel von Unterpunkt2

\item \subsection{Lösungserwartung:} 

\Subitem{$h''(t)=\frac{1}{4}\cdot(-t+9)$} %Lösung von Unterpunkt1
\Subitem{Die Kurve ist für $t<9$ linksgekrümmt, d.h., die Wachstumsgeschwindigkeit nimmt zu. Die Kurve ist für $t>9$ rechtsgekrümmt, d.h., die Wachstumsgeschwindigkeit nimmt ab. Daher ist die Wachstumsgeschwindigkeit nach 9 Wochen am größten. Die Pflanz wurde also am Beginn der 8. Woche gedüngt.
	
	Ein weiterer Lösungsansatz wäre, das Maximum der Wachstumsfunktion (also von $h'$) zu bestimmen.} %%Lösung von Unterpunkt2

\setcounter{subitemcounter}{0}
\subsection{Lösungsschlüssel:}
 
\Subitem{Ein Punkt für die korrekte Berechnung.} %Lösungschlüssel von Unterpunkt1
\Subitem{Ein Punkt für die korrekte Begründung.} %Lösungschlüssel von Unterpunkt2

\item \subsection{Lösungserwartung:} 

\Subitem{Die Funktion $h_1$ ist rechtsgekrümmt, die Funktion $h_2$ ist linksgekrümmt, das Krümmungsverhalten der Funktion $h$ ändert sich.} %Lösung von Unterpunkt1
\Subitem{Das bedeutet, die Wachstumsgeschwindigkeit derjenigen Pflanze, die durch $h_1$ beschrieben wird, wird immer kleiner (sie wächst immer langsamer) und die Wachstumsgeschwindigkeit derjenigen Pflanze, die durch $h_2$ beschrieben wird, wird immer größer (sie wächst immer schneller).
	
	Im Vergleich dazu ändert sich das Monotonieverhalten der Wachstumsgeschwindigkeit bei derjenigen Pflanze, die durch $h$ beschrieben wird, an der Stelle $t=9$ [vgl. c)].} %%Lösung von Unterpunkt2

\setcounter{subitemcounter}{0}
\subsection{Lösungsschlüssel:}
 
\Subitem{Ein Punkt für das korrekte Beschreiben des Krümmungsverhalten.} %Lösungschlüssel von Unterpunkt1
\Subitem{Ein Punkt für die richtige Interpretation.} %Lösungschlüssel von Unterpunkt2

\end{loesung}\end{langesbeispiel}

\newpage

%\begin{langesbeispiel} \item[8]	% Aufgabe: 10
%An einem gefällten Baum kann anhand der Jahresringe der jeweilige Umfang des Baumstamms zu einem bestimmten Baumalter ermittelt werden. Die Untersuchung eines Baumes ergab folgende Zusammenhänge zwischen Alter und Umfang:
%				
%				\resizebox{1\linewidth}{!}{\begin{tabular}{|l|c|c|c|c|c|c|c|c|}\hline
%				Alter $t$ (in Jahren)&25&50&75&100&125&150&175&200\\ \hline
%				Umfang $u$ (in Metern)&0,462&1,256&2,465&3,370&3,761&3,895&3,934&3,950\\ \hline				
%				\end{tabular}}
%				
%				Der Zusammenhang zwischen Alter und Umfang kann durch eine Wachstumsfunktion $u$ beschrieben werden, wobei der Wert $u(t)$ den Umfang zum Zeitpunkt $t$ angibt.
%				
%				In der nachstehenden Graphik sind die gemessenen Werte und der Graph der Wachstumsfunktion $u$ veranschaulicht.\vspace{0,2cm}
%				
%				\psset{xunit=0.06cm,yunit=2.0cm,algebraic=true,dimen=middle,dotstyle=o,dotsize=5pt 0,linewidth=0.8pt,arrowsize=3pt 2,arrowinset=0.25}
%\begin{pspicture*}(-14.67098559637483,-0.2288481488902294)(217.3773296649945,4.3033504752999825)
%\multips(0,0)(0,0.25){19}{\psline[linestyle=dashed,linecap=1,dash=1.5pt 1.5pt,linewidth=0.4pt,linecolor=gray]{c-c}(0,0)(217.3773296649945,0)}
%\multips(0,0)(25.0,0){10}{\psline[linestyle=dashed,linecap=1,dash=1.5pt 1.5pt,linewidth=0.4pt,linecolor=gray]{c-c}(0,0)(0,4.3033504752999825)}
%\psaxes[comma,labelFontSize=\scriptstyle,showorigin=false,xAxis=true,yAxis=true,Dx=25.,Dy=0.25,ticksize=-2pt 0,subticks=0]{->}(0,0)(-17.67098559637483,-0.1888481488902294)(217.3773296649945,4.3033504752999825)[$t$,140] [$u(t)$,-40]
%\psplot[linewidth=1.6pt,plotpoints=200]{0}{217.3773296649945}{3.9492122403178858/(1.0+26.648008639299004*EXP(-0.050467291467754964*x))}
%\rput[tl](80.12460592328863,3.1628008192524293){u}
%\begin{scriptsize}
%\psdots[dotsize=4pt 0,dotstyle=*](25.,0.462)
%\psdots[dotsize=4pt 0,dotstyle=*](50.,1.256)
%\psdots[dotsize=4pt 0,dotstyle=*](75.,2.465)
%\psdots[dotsize=4pt 0,dotstyle=*](100.,3.37)
%\psdots[dotsize=4pt 0,dotstyle=*](125.,3.761)
%\psdots[dotsize=4pt 0,dotstyle=*](150.,3.895)
%\psdots[dotsize=4pt 0,dotstyle=*](175.,3.934)
%\psdots[dotsize=4pt 0,dotstyle=*](200.,3.95)
%\end{scriptsize}
%\end{pspicture*}%Aufgabentext
%
%\begin{aufgabenstellung}
%\item Innerhalb der ersten 50 Jahre wird eine exponentielle Zunahme des Umfangs angenommen.%Aufgabentext
%
%\Subitem{Ermittle aus den Werten der Tabelle für 25 und 50 Jahre eine Wachstumsfunktion für diesen Zeitraum.} %Unterpunkt1
%\Subitem{Begründe mittels einer Rechnung, warum dieses Modell für die darauffolgenden 
%25 Jahre nicht mehr gilt.} %Unterpunkt2
%
%\item %Aufgabentext
%
%\Subitem{Berechne den Differenzenquotienten im Zeitintervall von 75 bis 100 Jahren. Gib an, was dieser Wert über das Wachstum des Baumes aussagt.} %Unterpunkt1
%\Subitem{Erläutere, was die 1. Ableitungsfunktion $u'$ im gegebenen Zusammenhang beschreibt} %Unterpunkt2
%
%\item %Aufgabentext
%
%\Subitem{Schätze mithilfe der Grafik denjenigen Zeitpunkt ab, zu dem der Umfang des Baumes am schnellsten zugenommen hat. Gib den Namen des charakteristischen Punktes des Graphen der Funktion an, der diesen Zeitpunkt bestimmt.} %Unterpunkt1
%\Subitem{Beschreibe, wie dieser Zeitpunkt rechnerisch ermittelt werden kann, wenn die Wachstumsfunktion $u$ bekannt ist.} %Unterpunkt2
%
%\item Die beiden Wachstumsfunktionen $f$ und $g$ mit $f(t)=a\cdot q^t$ und $g(t)=b\cdot e^{k\cdot t}$ beschreiben denselben Wachstumsprozess, sodass $f(t)=g(t)$ für alle $t$ gelten muss.%Aufgabentext
%
%\Subitem{Gib die Zusammenhänge zwischen den Parametern $a$ und $b$ beziehungsweise $q$ und $k$ jeweils in Form einer Gleichung an.} %Unterpunkt1
%\Subitem{Gib an, welche Werte die Parameter $q$ und $k$ annehmen können, wenn die Funktionen $f$ und $g$ im Zusammenhang mit einer exponentiellen Abnahme verwendet werden.} %Unterpunkt2
%
%\end{aufgabenstellung}
%
%\begin{loesung}
%\item \subsection{Lösungserwartung:} 
%
%\Subitem{$f(t)=a\cdot q^t \rightarrow 1,256=a\cdot q^{50}$ bzw. $0,462=a\cdot q^{25}$
%	
%	$\rightarrow$ (Division) $2,71861=q^{25} \rightarrow q\approx 1,0408 \rightarrow a=\frac{0,462}{q^{25}} \rightarrow a\approx 0,17$
%	
%	$\rightarrow$ (näherungsweise) $f(t)=0,17\cdot 1,0408^t$ bzw. $f(t)=0,17\cdot e^{0,04\cdot t}$ da $\ln(1,0408)\approx 0,04$} %Lösung von Unterpunkt1
%\Subitem{Begründung dafür, dass das Modell für die nächsten 25 Jahre nicht passend ist: Nach dem Modell gilt $f(75)=0,17\cdot 1,0408^{75}\approx 3,412$. Dieser Wert weicht signifikant vom gemessen Wert ab und spricht daher gegen eine Verwendung des exponentiellen Modells in den nächsten 25 Jahren.} %%Lösung von Unterpunkt2
%
%\setcounter{subitemcounter}{0}
%\subsection{Lösungsschlüssel:}
% 
%\Subitem{Ein Punkt für die Wachstumsfunktion.} %Lösungschlüssel von Unterpunkt1
%\Subitem{Ein Punkt für eine korrekte Begründung.} %Lösungschlüssel von Unterpunkt2
%
%\item \subsection{Lösungserwartung:} 
%
%\Subitem{Differenzenquotient: $\frac{3,370-2,465}{100-75}\approx 0,036$
%	
%	Die durchschnittliche Zunahme zwischen 75 und 100 Jahren beträgt 3,6 cm pro Jahr.} %Lösung von Unterpunkt1
%\Subitem{Die 1. Ableitungsfunktion gibt die momentane Wachstumsrate an.} %%Lösung von Unterpunkt2
%
%\setcounter{subitemcounter}{0}
%\subsection{Lösungsschlüssel:}
% 
%\Subitem{Ein Punkt für den richtigen Differenzenquotient sowie eine entsprechende Interpretation.} %Lösungschlüssel von Unterpunkt1
%\Subitem{Ein Punkt für eine korrekte Interpretation der 1. Ableitungsfunktion.} %Lösungschlüssel von Unterpunkt2
%
%\item \subsection{Lösungserwartung:} 
%
%\Subitem{Der charakteristische Punkt ist der Wendepunkt. Die Wendestelle der Funktion bestimmt den Zeitpunkt für das maximale jährliche Wachstum des Baumumfangs. Am schnellsten nimmt der Baum bei etwa 65 Jahren an Umfang zu (Lösungsintervall $[55;75]$).} %Lösung von Unterpunkt1
%\Subitem{Die Nullstelle der 2. Ableitungsfunktion bestimmt in diesem Fall denjenigen Zeitpunkt, zu dem der Baumumfang am schnellsten zunimmt.} %%Lösung von Unterpunkt2
%
%\setcounter{subitemcounter}{0}
%\subsection{Lösungsschlüssel:}
% 
%\Subitem{Ein Punkt für eine korrekte Schätzung sowie den passenden Namen des Punktes.} %Lösungschlüssel von Unterpunkt1
%\Subitem{Ein Punkt für eine korrekte Beschreibung.} %Lösungschlüssel von Unterpunkt2
%
%\item \subsection{Lösungserwartung:} 
%
%\Subitem{$f(0)=a$ und $g(0)=b$, daraus folgt: $a$ und $b$ sind gleich.
%	
%	Da $q^t=e^{k\cdot t}$ gilt, folgt $\ln(q)=k$ bzw. $q=e^k$.} %Lösung von Unterpunkt1
%\Subitem{Bei einer Zerfallsfunktion muss $0<q<1$ bzw. $k<0$ gelten.} %%Lösung von Unterpunkt2
%
%\setcounter{subitemcounter}{0}
%\subsection{Lösungsschlüssel:}
% 
%\Subitem{Ein Punkt für die richtigen Gleichungen.} %Lösungschlüssel von Unterpunkt1
%\Subitem{Ein Punkt für die richtigen Bedingungen.} %Lösungschlüssel von Unterpunkt2
%
%\end{loesung}\end{langesbeispiel}
%
%
\newpage

\begin{beispiel}{0} %PUNKTE DES BEISPIELS
Das ist ein Test

\tfmultiplechoice[5]{Statement}{  %Anzahl der Antwortmoeglichkeiten, Standard: 5
						L1={adlsak},   %1. Antwortmoeglichkeit 
						L2={dklask},   %2. Antwortmoeglichkeit
						L3={dklsa},   %3. Antwortmoeglichkeit
						L4={dksal},   %4. Antwortmoeglichkeit
						L5={dklsa},	 %5. Antwortmoeglichkeit
						L6={},	 %6. Antwortmoeglichkeit
						L7={},	 %7. Antwortmoeglichkeit
						L8={},	 %8. Antwortmoeglichkeit
						L9={},	 %9. Antwortmoeglichkeit
						%% LOESUNG: %%
						A1=0,  % 1. Antwort
						A2=0,	 % 2. Antwort
						A3=0,  % 3. Antwort
						A4=0,  % 4. Antwort
						A5=0,  % 5. Antwort
						}
\end{beispiel}
\null\notenschluessel{0.91}{0.8}{0.64}{0.5}

\end{document}

% Aufgabenliste: 4, 10