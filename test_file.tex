

\documentclass[a4paper,12pt]{article}
\usepackage{geometry}
\geometry{a4paper,left=18mm,right=18mm, top=2cm, bottom=2cm}
 

\usepackage{lmodern}
\usepackage[T1]{fontenc}
\usepackage[utf8]{inputenc}
\usepackage[ngerman]{babel}
\usepackage[solution_on, random=0, info_off]{srdp-mathematik} % solution_on/off, random, info_on/off


\pagestyle{plain} %PAGESTYLE: empty, plain
\onehalfspacing %Zeilenabstand
\setcounter{secnumdepth}{-1} % keine Nummerierung der Überschriften
%
%
%%%%%%%%%%%%%%%%%%%%%%%%%%%%%%%%%%%%%%%%%%%%%%%%%%%%%%%%%%%%%%%%%
%%%%%%%%%%%%%%%%%%%%%% DOKUMENT - ANFANG %%%%%%%%%%%%%%%%%%%%%%%%
%%%%%%%%%%%%%%%%%%%%%%%%%%%%%%%%%%%%%%%%%%%%%%%%%%%%%%%%%%%%%%%%%
%
%

\begin{document}

\begin{beispiel}[AG 1.1]{0} %PUNKTE DES BEISPIELS
Das ist ein Test

\langmultiplechoice[5]{  %Anzahl der Antwortmoeglichkeiten, Standard: 5
				L1={$\Vek{1}{3}{4} = \Vek{4}{4}{1} \cdot \Vek{2}{1}{x}$},   %1. Antwortmoeglichkeit 
				L2={•},   %2. Antwortmoeglichkeit
				L3={•},   %3. Antwortmoeglichkeit
				L4={•},   %4. Antwortmoeglichkeit
				L5={•},	 %5. Antwortmoeglichkeit
				L6={},	 %6. Antwortmoeglichkeit
				L7={},	 %7. Antwortmoeglichkeit
				L8={},	 %8. Antwortmoeglichkeit
				L9={},	 %9. Antwortmoeglichkeit
				%% LOESUNG: %%
				A1=1,  % 1. Antwort
				A2=5,	 % 2. Antwort
				A3=0,  % 3. Antwort
				A4=0,  % 4. Antwort
				A5=0,  % 5. Antwort
				}
\end{beispiel}
\end{document}

% Aufgabenliste: 4, 10