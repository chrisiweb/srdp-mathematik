\documentclass[a4paper,12pt]{article}
\usepackage{geometry}
\geometry{a4paper,left=18mm,right=18mm, top=2cm, bottom=2cm}
 

\usepackage{lmodern}
\usepackage[T1]{fontenc}
\usepackage[utf8]{inputenc}
\usepackage[ngerman]{babel}
\usepackage[solution_minimal, random=0, info_off]{srdp-mathematik} % solution_on/off, random, info_on/off
%
\usepackage{blindtext}
%
%\pagestyle{plain} %PAGESTYLE: empty, plain
%\onehalfspacing %Zeilenabstand
%\setcounter{secnumdepth}{-1} % keine Nummerierung der Überschriften
%
%
%%%%%%%%%%%%%%%%%%%%%%%%%%%%%%%%%%%%%%%%%%%%%%%%%%%%%%%%%%%%%%%%%
%%%%%%%%%%%%%%%%%%%%%% DOKUMENT - ANFANG %%%%%%%%%%%%1%%%%%%%%%%%%
%%%%%%%%%%%%%%%%%%%%%%%%%%%%%%%%%%%%%%%%%%%%%%%%%%%%%%%%%%%%%%%%%
%
%

%\ExplSyntaxOn
%\fp_new:N \punkteTotal
%\NewDocumentCommand{\punkteAddieren}{m}{
%  \fp_gadd:Nn \punkteTotal {#1}
%}
%\NewDocumentCommand{\punkteReset}{}{
%  \fp_zero:N \punkteTotal 
%}
%\NewDocumentCommand{\punkteDisplay}{}{
%  \fp_to_decimal:N \punkteTotal
%}
%
%
%\ExplSyntaxOff


\begin{document}


\smallskip\begin{minipage}{1\textwidth}
{\color{green!40!black!60!}{\fbox{Anzahl weiterer Variationen dieser Aufgabe: 3}}}\vspace{-0.5cm}

\subsection{AG 1.2 - 1 - Oberfläche eines Zylinders}\smallskip

\end{minipage}

\begin{beispiel}[AG 1.2]{1}
Für die Oberfläche $O$ eines Zylinders mit dem Radius $r$ und der Höhe $h$ gilt $O=2r^2\pi+2r\pi h$.
	
	Welche der folgenden Aussagen sind im Zusammenhang mit der gegebenen Formel zutreffend? Kreuze die zutreffende(n) Aussage(n) an!
	\multiplechoice[5]{  %Anzahl der Antwortmoeglichkeiten, Standard: 5
	L1={$O>2r^2\pi + r\pi h$ ist eine Formel.},   %1. Antwortmoeglichkeit 
	L2={$2r^2\pi + 2r\pi h$ ist ein Term.},   %2. Antwortmoeglichkeit
	L3={Jede Variable ist ein Term.},   %3. Antwortmoeglichkeit
	L4={$O=2r\pi \cdot \left(r+h\right)$ entsteht durch Umformung aus $O=2r^2\pi + 2r\pi h$.},   %4. Antwortmoeglichkeit
	L5={$\pi$ ist eine Variable.},	 %5. Antwortmoeglichkeit
	L6={},	 %6. Antwortmoeglichkeit
	L7={},	 %7. Antwortmoeglichkeit
	L8={},	 %8. Antwortmoeglichkeit
	L9={},	 %9. Antwortmoeglichkeit
	%% LOESUNG: %%
	A1=2,  % 1. Antwort
	A2=3,	 % 2. Antwort
	A3=4,  % 3. Antwort
	A4=0,  % 4. Antwort
	A5=0,  % 5. Antwort
	}
\end{beispiel}

\info{\fbox{\begin{minipage}{0.98\textwidth}
Titel: Oberfläche eines Zylinders\\
Grundkompetenz(en): AG 1.2\\
Aufgabenformat: Mutiple Choice\\
Klasse: 5\\
Quelle: BIFIE
\end{minipage}}}

\newpage

\smallskip\begin{minipage}{1\textwidth}
{\color{green!40!black!60!}{\fbox{Anzahl weiterer Variationen dieser Aufgabe: 2}}}\vspace{-0.5cm}

\subsection{AG 1.2 - 2 - Äquivalenz}\smallskip

\end{minipage}

\begin{beispiel}[AG 1.2]{1}
Gegeben ist der Term $\frac{x}{2b}-\frac{y}{b}$ mit $b\neq 0$.
			
			Kreuze den/die zum gegebenen Term äquivalenten Term(e) an!
			\multiplechoice[5]{  %Anzahl der Antwortmoeglichkeiten, Standard: 5
	L1={$\frac{2x-y}{2b}$},   %1. Antwortmoeglichkeit 
	L2={$\frac{x-2y}{b}$},   %2. Antwortmoeglichkeit
	L3={$\frac{x-2y}{2b}$},   %3. Antwortmoeglichkeit
	L4={$\frac{x-y}{b}$},   %4. Antwortmoeglichkeit
	L5={$x-2y:2b$},	 %5. Antwortmoeglichkeit
	L6={},	 %6. Antwortmoeglichkeit
	L7={},	 %7. Antwortmoeglichkeit
	L8={},	 %8. Antwortmoeglichkeit
	L9={},	 %9. Antwortmoeglichkeit
	%% LOESUNG: %%
	A1=3,  % 1. Antwort
	A2=0,	 % 2. Antwort
	A3=0,  % 3. Antwort
	A4=0,  % 4. Antwort
	A5=0,  % 5. Antwort
	}
\end{beispiel}

\info{\fbox{\begin{minipage}{0.98\textwidth}
Titel: Äquivalenz\\
Grundkompetenz(en): AG 1.2\\
Aufgabenformat: Mutiple Choice\\
Klasse: 5\\
Quelle: BIFIE
\end{minipage}}}

\newpage

\smallskip\begin{minipage}{1\textwidth}
{\color{green!40!black!60!}{\fbox{Anzahl weiterer Variationen dieser Aufgabe: 2}}}\vspace{-0.5cm}

\subsection{AG 1.2 - 3 - Rationale Exponenten}\smallskip

\end{minipage}

\begin{beispiel}[AG 1.2]{1}
Welche der angeführten Terme sind äquivalent zum Term $x^{\frac{5}{3}}$ $\left( \text{mit } x>0 \right)$?
			
			Kreuze die beiden zutreffenden Terme an!
			\multiplechoice[5]{  %Anzahl der Antwortmoeglichkeiten, Standard: 5
	L1={$\frac{1}{x^{\frac{5}{3}}}$},   %1. Antwortmoeglichkeit 
	L2={$\sqrt[3]{x^{5}}$},   %2. Antwortmoeglichkeit
	L3={$x^{-\frac{3}{5}}$},   %3. Antwortmoeglichkeit
	L4={$\sqrt[5]{x^3}$},   %4. Antwortmoeglichkeit
	L5={$x\cdot \sqrt[3]{x^2}$},	 %5. Antwortmoeglichkeit
	L6={},	 %6. Antwortmoeglichkeit
	L7={},	 %7. Antwortmoeglichkeit
	L8={},	 %8. Antwortmoeglichkeit
	L9={},	 %9. Antwortmoeglichkeit
	%% LOESUNG: %%
	A1=2,  % 1. Antwort
	A2=5,	 % 2. Antwort
	A3=0,  % 3. Antwort
	A4=0,  % 4. Antwort
	A5=0,  % 5. Antwort
	}
\end{beispiel}

\info{\fbox{\begin{minipage}{0.98\textwidth}
Titel: Rationale Exponenten\\
Grundkompetenz(en): AG 1.2\\
Aufgabenformat: Mutiple Choice\\
Klasse: 6\\
Quelle: BIFIE
\end{minipage}}}

\newpage

\smallskip\begin{minipage}{1\textwidth}
 \flagUK\ \vspace{-0.5cm}

\subsection{AG 1.2 - 4 - Äquivalenzumformung (Matura Haupttermin 15/16)}\smallskip

\end{minipage}

\begin{beispiel}[AG 1.2]{1}
Nicht jede Umformung einer Gleichung ist eine Äquivalenzumformung.

Erkläre konkret auf das unten angegebene Beispiel bezogen, warum es sich bei der durchgeführten
Umformung um keine Äquivalenzumformung handelt! Die Grundmenge ist die Menge
der reellen Zahlen.

\begin{align*}
x^2 - 5x &= 0 \qquad |:x\\
x-5 &= 0
\end{align*}

\antwort{Mögliche Erklärung: \\
Die Gleichung $x^2 - 5x = 0$ hat die Lösungen $x_1 = 5$ und $x_2 = 0$ (die Lösungsmenge $L = \{0; ~5\}$). Die Gleichung $x - 5 = 0$ hat aber nur mehr die Lösung $x = 5$ (die Lösungsmenge $L = \{5\}$). Durch die durchgeführte Umformung wurde die Lösungsmenge verändert, daher ist dies keine Äquivalenzumformung. \leer

ODER: \leer

Bei der Division durch $x$ würde im Fall $x = 0$ durch null dividiert werden, was keine zulässige Rechenoperation ist.}
\end{beispiel}

\info{\fbox{\begin{minipage}{0.98\textwidth}
Titel: Äquivalenzumformung\\
Grundkompetenz(en): AG 1.2\\
Aufgabenformat: Offenes Antwortformat\\
Klasse: 5\\
Quelle: Matura Haupttermin 15/16
\end{minipage}}}

\newpage

\smallskip\begin{minipage}{1\textwidth}
\subsection{AG 1.2 - 5 - Punktladungen (Matura Haupttermin 13/14)}\smallskip

\end{minipage}

\begin{beispiel}[AG 1.2]{1}
Der Betrag $F$ der Kraft zwischen Punktladungen $q_1$ und $q_2$ im Abstand $r$ wird beschrieben durch die Gleichung $F=C\cdot \frac{q_1\cdot q_2}{r^2}$ ($C$ ... physikalische Konstante).
	
	Gib an, um welchen Faktor sich der Betrag $F$ ändert, wenn der Betrag der Punktlandungen $q_1$ und $q_2$ jeweils verdoppelt und der Abstand $r$ zwischen diesen beiden Punktlandungen halbiert wird.\leer
	
\antwort{$F=C\cdot\dfrac{2\cdot q_1\cdot 2\cdot q_2}{\left(\frac{r}{2}\right)^2}=C\cdot \dfrac{16\cdot q_1\cdot q_2}{r^2}$

Der Betrag der Kraft $F$ wird 16-mal so groß.

\textit{Ein Punkt für die richtige Lösung. Weder die Rechnung noch ein Antwortsatz müssen angegeben werden. Die Angabe des Faktors 16 ist ausreichend.}}
\end{beispiel}

\info{\fbox{\begin{minipage}{0.98\textwidth}
Titel: Punktladungen\\
Grundkompetenz(en): AG 1.2\\
Aufgabenformat: Offenes Antwortformat\\
Klasse: 6\\
Quelle: Matura Haupttermin 13/14
\end{minipage}}}

\newpage

\smallskip\begin{minipage}{1\textwidth}
\subsection{AG 1.2 - 6 - Definitionsmengen (Matura Herbsttermin 13/14)}\smallskip

\end{minipage}

\begin{beispiel}[AG 1.2]{1}
Es sind vier Terme und sechs Mengen (A bis F) gegeben.
	
	Ordne den vier Termen jeweils die entsprechende größtmögliche Definitionsmenge $D_A, D_B, ... , D_F$ in der Menge der reellen Zahlen zu!\leer
	
	\zuordnen{
	R1={$\ln(x+1)$},	% Response 1
	R2={$\sqrt{1-x}$},	% Response 2
	R3={$\frac{2x}{x\cdot(x+1)^2}$},	% Response 3
	R4={$\frac{2x}{x^2+1}$},	% Response 4
	%% Moegliche Zuordnungen: %%
	A={$D_A=\mathbb{R}$}, 	%Moeglichkeit A  
	B={$D_B=(1;\infty)$}, 	%Moeglichkeit B  
	C={$D_C=(-1;\infty)$}, 	%Moeglichkeit C  
	D={$D_D=\mathbb{R}\backslash\left\{-1;0\right\}$}, 	%Moeglichkeit D  
	E={$D_E=(-\infty;1)$}, 	%Moeglichkeit E  
	F={$D_F=(-\infty;1]$}, 	%Moeglichkeit F  
	%% LOESUNG: %%
	A1={C},	% 1. richtige Zuordnung
	A2={F},	% 2. richtige Zuordnung
	A3={D},	% 3. richtige Zuordnung
	A4={A},	% 4. richtige Zuordnung
	}
\end{beispiel}

\info{\fbox{\begin{minipage}{0.98\textwidth}
Titel: Definitionsmengen\\
Grundkompetenz(en): AG 1.2\\
Aufgabenformat: Zuordnungsformat\\
Klasse: 6\\
Quelle: Matura Herbsttermin 13/14
\end{minipage}}}

\newpage

\smallskip\begin{minipage}{1\textwidth}
 \flagUK\ {\color{green!40!black!60!}{\fbox{Anzahl weiterer Variationen dieser Aufgabe: 2}}}\vspace{-0.5cm}

\subsection{AG 1.2 - 7 - Äquivalente Gleichungen (Matura Wintertermin 18/19)}\smallskip

\end{minipage}

\begin{beispiel}[AG 1.2]{1}
Gegeben ist die Gleichung $\frac{x}{2}-4=3$ in $x\in\mathbb{R}$.

Kreuze die beiden nachstehenden Gleichungen $x\in\mathbb{R}$ an, die zur gegebenen Gleichung äquivalent sind.

\multiplechoice[5]{  %Anzahl der Antwortmoeglichkeiten, Standard: 5
	L1={$x-4=6$},   %1. Antwortmoeglichkeit 
	L2={$\frac{x}{2}=-1$},   %2. Antwortmoeglichkeit
	L3={$\frac{x}{2}-3=4$},   %3. Antwortmoeglichkeit
	L4={$\frac{x-8}{2}=3$},   %4. Antwortmoeglichkeit
	L5={$\left(\frac{x}{2}-4\right)^2=9$},	 %5. Antwortmoeglichkeit
	L6={},	 %6. Antwortmoeglichkeit
	L7={},	 %7. Antwortmoeglichkeit
	L8={},	 %8. Antwortmoeglichkeit
	L9={},	 %9. Antwortmoeglichkeit
	%% LOESUNG: %%
	A1=3,  % 1. Antwort
	A2=4,	 % 2. Antwort
	A3=0,  % 3. Antwort
	A4=0,  % 4. Antwort
	A5=0,  % 5. Antwort
	}
\end{beispiel}

\info{\fbox{\begin{minipage}{0.98\textwidth}
Titel: Äquivalente Gleichungen\\
Grundkompetenz(en): AG 1.2\\
Aufgabenformat: Mutiple Choice\\
Klasse: 5\\
Quelle: Matura Wintertermin 18/19
\end{minipage}}}

\newpage

\smallskip\begin{minipage}{1\textwidth}
\subsection{AG 1.2 - 8 - Umformen und einsetzen in Formeln}\smallskip

\end{minipage}

\begin{beispiel}[AG 1.2]{1}
Gegeben ist folgende Schaltung:

\newrgbcolor{ttqqtt}{0.2 0 0.2}
\psset{xunit=0.65cm,yunit=0.65cm,algebraic=true,dimen=middle,dotstyle=o,dotsize=5pt 0,linewidth=1.6pt,arrowsize=5pt 2,arrowinset=0.25}
\begin{center}
\begin{pspicture*}(1.65,2.8)(9.2,7.3)
\psaxes[labelFontSize=\scriptstyle,xAxis=true,yAxis=true,Dx=1,Dy=1,ticksize=-2pt 0,subticks=2]{->}(0,0)(-2.2534905675922268,-3.8839004709576583)(14.463571792619675,10.543358360953654)
\psline[linewidth=1.7pt](2,3)(2,6)
\psline[linewidth=1.7pt](2,6)(3,6)
\psline[linewidth=1.7pt](3,6.5)(5,6.5)
\psline[linewidth=1.7pt](3,5.5)(5,5.5)
\psline[linewidth=1.7pt](5,6.5)(5,5.5)
\psline[linewidth=1.7pt](3,5.5)(3,6.5)
\psline[linewidth=1.7pt](5,6)(6,6)
\psline[linewidth=1.7pt](6,6.5)(6,5.5)
\psline[linewidth=1.7pt](6,6.5)(8,6.5)
\psline[linewidth=1.7pt](8,5.5)(8,6.5)
\psline[linewidth=1.7pt](6,5.5)(8,5.5)
\psline[linewidth=1.7pt](8,6)(9,6)
\psline[linewidth=1.7pt](9,6)(9,3)
\psline[linewidth=1pt]{->}(3,4.5)(5,4.5)
\psline[linewidth=1pt]{->}(6,4.5)(8,4.5)
\psline[linewidth=1pt]{->}(2.2,3)(8.8,3)
\psline[linewidth=1pt]{->}(2,6)(2.7301994645926997,6)
\psline[linewidth=1pt]{->}(5,6)(5.7,6)
\begin{scriptsize}
\psdots[dotstyle=*,linecolor=ttqqtt](2,3)
\rput[bl](3.6,6.7){\normalsize $R_1$}
\rput[bl](6.6,6.7){\normalsize $R_2$}
\psdots[dotstyle=*,linecolor=ttqqtt](9,3)
\rput[bl](3.6,4.7){\normalsize $U_1$}
\rput[bl](6.6,4.7){\normalsize $U_2$}
\rput[bl](5.0,3.2){\normalsize $U_{ges}$}
\rput[bl](2.2,6.25){\normalsize $I_1$}
\rput[bl](5.15,6.25){\normalsize $I_2$}
\end{scriptsize}
\end{pspicture*}
\end{center}


$R_1$ und $R_2$ sind die elektrischen Widerstände. $U_1$, $U_2$ und $U_{ges}$ sind die elektrischen Spannungen. $I_1$ und $I_2$ sind die elektrischen Stromstärken. 

Für die Gesamtspannung gilt: $U_{ges}=U_1+U_2$. \\
Für die elektrischen Stromstärken gilt: $I_1 = I_2$.

Kreuze die beiden zutreffenden Aussagen an! 

\multiplechoice[5]{  %Anzahl der Antwortmoeglichkeiten, Standard: 5
	L1={Wenn man die Differenz von den elektrischen Stromstärken $I_1$ und $I_2$ berechnet, dann ist das Ergebnis größer als 0.
	},   %1. Antwortmoeglichkeit 
	L2={Wenn der elektrische Widerstand durch $R_1= \dfrac{U_1}{I_1}$ berechnet wird, dann kann er auch durch $R_1= \dfrac{U_{ges}+U_2}{I_1}$ bestimmt werden.},   %2. Antwortmoeglichkeit
	L3={Wenn die elektrischen Stromstärken durch $I_1=\dfrac{U_1}{R_1}$ und $I_2=\dfrac{U_2}{R_2}$ gegeben sind, dann gilt die Proportion $\dfrac{U_1}{R_1}=\dfrac{U_2}{R_2}$. },   %3. Antwortmoeglichkeit
	L4={ Wenn man die Gleichung $\dfrac{U_{ges} - U_2}{R_1}= \dfrac{U_2}{R_2}$ nach $U_{ges}$ umformt, dann erhält man $U_{ges}= \dfrac{U_2 \cdot R_1}{R_2} - U_2$.},   %4. Antwortmoeglichkeit
	L5={Wenn die elektrischen Spannungen $U_1$ und $U_2$ gleich groß sind, dann ergibt $U_2$ die Hälfte der Gesamtspannung $U_{ges}.$},	 %5. Antwortmoeglichkeit
	L6={},	 %6. Antwortmoeglichkeit
	L7={},	 %7. Antwortmoeglichkeit
	L8={},	 %8. Antwortmoeglichkeit
	L9={},	 %9. Antwortmoeglichkeit
	%% LOESUNG: %%
	A1=3,  % 1. Antwort
	A2=5,	 % 2. Antwort
	A3=0,  % 3. Antwort
	A4=0,  % 4. Antwort
	A5=0,  % 5. Antwort
	}
\end{beispiel}

\info{\fbox{\begin{minipage}{0.98\textwidth}
Titel: Umformen und einsetzen in Formeln\\
Grundkompetenz(en): AG 1.2\\
Aufgabenformat: Mutiple Choice\\
Klasse: ?\\
Quelle: FilJan
\end{minipage}}}

\newpage

\smallskip\begin{minipage}{1\textwidth}
 \flagUK\ {\color{green!40!black!60!}{\fbox{Anzahl weiterer Variationen dieser Aufgabe: 2}}}\vspace{-0.5cm}

\subsection{AG 1.2 - 9 - Lösung einer Gleichung (Matura Wintertermin 19/20)}\smallskip

\end{minipage}

\begin{beispiel}[AG 1.2]{1}
Nachstehend ist eine Gleichung in $x\in\mathbb{R}$ gegeben.

$\sqrt{2\cdot x-6}=a$ mit $a\in\mathbb{R}^+_0$

Kreuze dasjenige Intervall an, das für alle Werte von $a\in\mathbb{R}^+_0$ die Lösung der gegebenen Gleichung enthält.

\multiplechoice[6]{  %Anzahl der Antwortmoeglichkeiten, Standard: 5
	L1={$(-\infty;-3]$},   %1. Antwortmoeglichkeit 
	L2={$[3;\infty)$},   %2. Antwortmoeglichkeit
	L3={$[-3;0)$},   %3. Antwortmoeglichkeit
	L4={$[0;3)$},   %4. Antwortmoeglichkeit
	L5={$[-6;-3)$},	 %5. Antwortmoeglichkeit
	L6={$[3;6]$},	 %6. Antwortmoeglichkeit
	L7={},	 %7. Antwortmoeglichkeit
	L8={},	 %8. Antwortmoeglichkeit
	L9={},	 %9. Antwortmoeglichkeit
	%% LOESUNG: %%
	A1=2,  % 1. Antwort
	A2=0,	 % 2. Antwort
	A3=0,  % 3. Antwort
	A4=0,  % 4. Antwort
	A5=0,  % 5. Antwort
	}
\end{beispiel}

\info{\fbox{\begin{minipage}{0.98\textwidth}
Titel: Lösung einer Gleichung\\
Grundkompetenz(en): AG 1.2\\
Aufgabenformat: Mutiple Choice\\
Klasse: 5\\
Quelle: Matura Wintertermin 19/20
\end{minipage}}}

\newpage

\smallskip\begin{minipage}{1\textwidth}
{\color{green!40!black!60!}{\fbox{Anzahl weiterer Variationen dieser Aufgabe: 2}}}\vspace{-0.5cm}

\subsection{AG 1.2 - 10 - Herausheben}\smallskip

\end{minipage}

\begin{beispiel}[AG 1.2]{1}
Gegeben ist der Term $b^3-2b^3+4b^2-10$. 
	
	Kreuze jene beiden Terme an, die zum gegebenen Term äquivalent sind.\vspace{0,2cm}

\multiplechoice[5]{  %Anzahl der Antwortmoeglichkeiten, Standard: 5
	L1={$-b\cdot(b^2+2b^2-4b)-10$},   %1. Antwortmoeglichkeit 
	L2={$b^2\cdot(b-2b+4)-10$},   %2. Antwortmoeglichkeit
	L3={$3b^5-10$},   %3. Antwortmoeglichkeit
	L4={$b\cdot(-b^2+4b)-10$},   %4. Antwortmoeglichkeit
	L5={$b\cdot(b^2-2b^2+4b-10)$},	 %5. Antwortmoeglichkeit
	L6={},	 %6. Antwortmoeglichkeit
	L7={},	 %7. Antwortmoeglichkeit
	L8={},	 %8. Antwortmoeglichkeit
	L9={},	 %9. Antwortmoeglichkeit
	%% LOESUNG: %%
	A1=2,  % 1. Antwort
	A2=4,	 % 2. Antwort
	A3=0,  % 3. Antwort
	A4=0,  % 4. Antwort
	A5=0,  % 5. Antwort
	}
\end{beispiel}

\info{\fbox{\begin{minipage}{0.98\textwidth}
Titel: Herausheben\\
Grundkompetenz(en): AG 1.2\\
Aufgabenformat: Mutiple Choice\\
Klasse: 5\\
Quelle: LaMA
\end{minipage}}}

\newpage

\smallskip\begin{minipage}{1\textwidth}
{\color{green!40!black!60!}{\fbox{Anzahl weiterer Variationen dieser Aufgabe: 1}}}\vspace{-0.5cm}

\subsection{AG 1.2 - 11 - Rationale Exponenten}\smallskip

\end{minipage}

\begin{beispiel}[AG 1.2]{1}
Ordne jedem Term den äquivalenten Term zu.

\zuordnen{
        R1={$x^{-\frac{4}{5}}$},        % Response 1
        R2={$\frac{4}{5}x$},    % Response 2
        R3={$x^{-\frac{5}{4}}$},        % Response 3
        R4={$x^{\frac{4}{5}}$}, % Response 4
        %% Moegliche Zuordnungen: %%
        A={$-\sqrt[5]{x^4}$},   %Moeglichkeit A 
        B={$\dfrac{1}{\sqrt[4]{x^{5}}}$},       %Moeglichkeit B 
        C={$x\cdot \sqrt[5]{x}$},       %Moeglichkeit C 
        D={$\frac{4x}{5}$},     %Moeglichkeit D 
        E={$\sqrt[5]{x^4}$},    %Moeglichkeit E 
        F={$\dfrac{1}{\sqrt[5]{x^4}}$},         %Moeglichkeit F 
        %% LOESUNG: %%
        A1={F}, % 1. richtige Zuordnung
        A2={D}, % 2. richtige Zuordnung
        A3={B}, % 3. richtige Zuordnung
        A4={E}, % 4. richtige Zuordnung
        }
\end{beispiel}

\info{\fbox{\begin{minipage}{0.98\textwidth}
Titel: Rationale Exponenten\\
Grundkompetenz(en): AG 1.2\\
Aufgabenformat: Zuordnungsformat\\
Klasse: 6\\
Quelle: LaMA
\end{minipage}}}

\newpage

\smallskip\begin{minipage}{1\textwidth}
{\color{green!40!black!60!}{\fbox{Anzahl weiterer Variationen dieser Aufgabe: 2}}}\vspace{-0.5cm}

\subsection{AG 1.2 - 13 - Termdefinition}\smallskip

\end{minipage}

\begin{beispiel}[AG 1.2]{1}
Gegeben ist der Term $T(x)=\dfrac{5x^2+3}{2\cdot x+3}$. 
	
	Gib jenen Wert für $x$ an, für den dieser Term nicht definiert ist.\vspace{0,5cm}

$x=$ \antwort[\rule{5cm}{0.3pt}]{$-\frac{3}{2}$}
\end{beispiel}

\info{\fbox{\begin{minipage}{0.98\textwidth}
Titel: Termdefinition\\
Grundkompetenz(en): AG 1.2\\
Aufgabenformat: Offenes Antwortformat\\
Klasse: 5\\
Quelle: LaMA
\end{minipage}}}

\newpage

\smallskip\begin{minipage}{1\textwidth}
{\color{green!40!black!60!}{\fbox{Anzahl weiterer Variationen dieser Aufgabe: 2}}}\vspace{-0.5cm}

\subsection{AG 1.2 - 14 - Formeln umformen}\smallskip

\end{minipage}

\begin{beispiel}[AG 1.2]{1}
Gegeben ist die Formel $K=bu+ur-d$. Forme die Formel auf $u=...$ um. Kreuze die zutreffende Formel an.\vspace{0,2cm}

\multiplechoice[6]{  %Anzahl der Antwortmoeglichkeiten, Standard: 5
	L1={$u=K+\frac{d}{b+r}$},   %1. Antwortmoeglichkeit 
	L2={$u=\frac{K-d}{b+r}$},   %2. Antwortmoeglichkeit
	L3={$u=K-\frac{d}{b+r}$},   %3. Antwortmoeglichkeit
	L4={$u=K-b-u+d$},   %4. Antwortmoeglichkeit
	L5={$u=\frac{K+d}{b+r}$},	 %5. Antwortmoeglichkeit
	L6={$u=\frac{K}{u+r}+d$},	 %6. Antwortmoeglichkeit
	L7={},	 %7. Antwortmoeglichkeit
	L8={},	 %8. Antwortmoeglichkeit
	L9={},	 %9. Antwortmoeglichkeit
	%% LOESUNG: %%
	A1=5,  % 1. Antwort
	A2=0,	 % 2. Antwort
	A3=0,  % 3. Antwort
	A4=0,  % 4. Antwort
	A5=0,  % 5. Antwort
	}
\end{beispiel}

\info{\fbox{\begin{minipage}{0.98\textwidth}
Titel: Formeln umformen\\
Grundkompetenz(en): AG 1.2\\
Aufgabenformat: Mutiple Choice\\
Klasse: 5\\
Quelle: LaMA
\end{minipage}}}

\newpage

\smallskip\begin{minipage}{1\textwidth}
{\color{green!40!black!60!}{\fbox{Anzahl weiterer Variationen dieser Aufgabe: 2}}}\vspace{-0.5cm}

\subsection{AG 1.2 - 15 - Äquivalenzumformungen}\smallskip

\end{minipage}

\begin{beispiel}[AG 1.2]{1}
Gegeben ist die Gleichung $-7=4x-\frac{6x}{11}$.\\
	\lueckentext{
	text={Wendet man die Äquivalenzumformung \gap an, dann erhält man \gap.}, 	%Lueckentext Luecke=\gap
	L1={$-7$}, 		%1.Moeglichkeit links  
	L2={$\cdot 11$}, 		%2.Moeglichkeit links
	L3={$:4$}, 		%3.Moeglichkeit links
	R1={$0=4x-\frac{6x}{11}-7$}, 		%1.Moeglichkeit rechts 
	R2={$-7=x-\frac{6x}{44}$}, 		%2.Moeglichkeit rechts
	R3={$-77=44x-6x$}, 		%3.Moeglichkeit rechts
	%% LOESUNG: %%
	A1=2,   % Antwort links
	A2=3		% Antwort rechts 
	}
\end{beispiel}

\info{\fbox{\begin{minipage}{0.98\textwidth}
Titel: Äquivalenzumformungen\\
Grundkompetenz(en): AG 1.2\\
Aufgabenformat: Lückentext\\
Klasse: 5\\
Quelle: LaMA
\end{minipage}}}

\newpage

\smallskip\begin{minipage}{1\textwidth}
{\color{green!40!black!60!}{\fbox{Anzahl weiterer Variationen dieser Aufgabe: 1}}}\vspace{-0.5cm}

\subsection{AG 1.2 - 16 - Rationale Exponenten}\smallskip

\end{minipage}

\begin{beispiel}[AG 1.2]{1}
Ordne jedem Term einen äquivalenten Term zu.\vspace{0,2cm}

\zuordnen{
	R1={$\dfrac{-4}{x^{-4}}$},	% Response 1
	R2={$\dfrac{-1}{x^4}$},	% Response 2
	R3={$\dfrac{-x^4}{4}$},	% Response 3
	R4={$x^{-4}$},	% Response 4
	%% Moegliche Zuordnungen: %%
	A={$-4x^{-4}$}, 	%Moeglichkeit A  
	B={$-x^4$}, 	%Moeglichkeit B  
	C={$\dfrac{1}{x^4}$}, 	%Moeglichkeit C  
	D={$-x^{-4}$}, 	%Moeglichkeit D  
	E={$\dfrac{x^4}{-4}$}, 	%Moeglichkeit E  
	F={$-4x^4$}, 	%Moeglichkeit F  
	%% LOESUNG: %%
	A1={F},	% 1. richtige Zuordnung
	A2={D},	% 2. richtige Zuordnung
	A3={E},	% 3. richtige Zuordnung
	A4={C},	% 4. richtige Zuordnung
	}
\end{beispiel}

\info{\fbox{\begin{minipage}{0.98\textwidth}
Titel: Rationale Exponenten\\
Grundkompetenz(en): AG 1.2\\
Aufgabenformat: Zuordnungsformat\\
Klasse: 6\\
Quelle: LaMA
\end{minipage}}}

\newpage

\smallskip\begin{minipage}{1\textwidth}
\subsection{AG 1.2 - 17 - Terme}\smallskip

\end{minipage}

\begin{beispiel}[AG 1.2]{1}
Gegeben ist die Definitionsmenge $D_T$ des Terms $T = \dfrac{1 - x}{x^2 + 5x}$. 

Kreuze die zutreffende Aussage an:

\multiplechoice[6]{  %Anzahl der Antwortmoeglichkeiten, Standard: 5
	L1={$D_T = \mathbb{R} \backslash \{1\}$},   %1. Antwortmoeglichkeit 
	L2={$D_T = \mathbb{R} \backslash \{1; 5\}$},   %2. Antwortmoeglichkeit
	L3={$D_T = \mathbb{R} \backslash \{0; 5\}$},   %3. Antwortmoeglichkeit
	L4={$D_T = \mathbb{R} \backslash \{1; -5\}$},   %4. Antwortmoeglichkeit
	L5={$D_T = \mathbb{R} \backslash \{0; -5\}$},	 %5. Antwortmoeglichkeit
	L6={$D_T = \mathbb{R} \backslash \{0; 1\}$},	 %6. Antwortmoeglichkeit
	L7={?},	 %7. Antwortmoeglichkeit
	L8={?},	 %8. Antwortmoeglichkeit
	L9={?},	 %9. Antwortmoeglichkeit
	%% LOESUNG: %%
	A1=5,  % 1. Antwort
	A2=0,	 % 2. Antwort
	A3=0,  % 3. Antwort
	A4=0,  % 4. Antwort
	A5=0,  % 5. Antwort
	}
\end{beispiel}

\info{\fbox{\begin{minipage}{0.98\textwidth}
Titel: Terme\\
Grundkompetenz(en): AG 1.2\\
Aufgabenformat: Mutiple Choice\\
Klasse: 5\\
Quelle: Thema Mathematik Schularbeiten 5. Klasse
\end{minipage}}}

\newpage

\smallskip\begin{minipage}{1\textwidth}
 \flagUK\ \vspace{-0.5cm}

\subsection{AG 1.2 - 18 - Differenz zwischen zwei natürlichen Zahlen (Matura Herbsttermin 20/21)}\smallskip

\end{minipage}

\begin{beispiel}[AG 1.2]{1}
Für zwei natürliche Zahlen $n$ und $m$ gilt: $n\neq m$.

Damit die Differenz $n-m$ eine natürliche Zahl ist, muss eine bestimmte mathematische Beziehung zwischen $n$ und $m$ gelten.

Gib diese mathematische Beziehung an.

\antwort{$n>m$ bzw. $n\geq m$}
\end{beispiel}

\info{\fbox{\begin{minipage}{0.98\textwidth}
Titel: Differenz zwischen zwei natürlichen Zahlen\\
Grundkompetenz(en): AG 1.2\\
Aufgabenformat: Offenes Antwortformat\\
Klasse: 5\\
Quelle: Matura Herbsttermin 20/21
\end{minipage}}}

\newpage

\smallskip\begin{minipage}{1\textwidth}
 \flagUK\ \vspace{-0.5cm}

\subsection{AG 1.2 - 19 - Werte von Termen (Matura Haupttermin 21/22)}\smallskip

\end{minipage}

\begin{beispiel}[AG 1.2]{1}
Nachstehend sind fünf Terme mit $a\in\mathbb{R}$ und $a<0$ gegeben.

Kreuze die beiden Terme an, deren Wert auf jeden Fall positiv ist.

\multiplechoice[5]{  %Anzahl der Antwortmoeglichkeiten, Standard: 5
				L1={$\dfrac{a-1}{a}$},   %1. Antwortmoeglichkeit 
				L2={$\dfrac{1-2\cdot a}{a}$},   %2. Antwortmoeglichkeit
				L3={$\dfrac{a}{1-a}$},   %3. Antwortmoeglichkeit
				L4={$a^2-1$},   %4. Antwortmoeglichkeit
				L5={$-a$},	 %5. Antwortmoeglichkeit
				L6={},	 %6. Antwortmoeglichkeit
				L7={},	 %7. Antwortmoeglichkeit
				L8={},	 %8. Antwortmoeglichkeit
				L9={},	 %9. Antwortmoeglichkeit
				%% LOESUNG: %%
				A1=1,  % 1. Antwort
				A2=5,	 % 2. Antwort
				A3=0,  % 3. Antwort
				A4=0,  % 4. Antwort
				A5=0,  % 5. Antwort
				}
\end{beispiel}

\info{\fbox{\begin{minipage}{0.98\textwidth}
Titel: Werte von Termen\\
Grundkompetenz(en): AG 1.2\\
Aufgabenformat: Mutiple Choice\\
Klasse: 5\\
Quelle: Matura Haupttermin 21/22
\end{minipage}}}

\newpage

\smallskip\begin{minipage}{1\textwidth}
\subsection{AG 1.2 - 20 - Definitonsmeng Wurzelgleichung}\smallskip

\end{minipage}

\begin{beispiel}[AG 1.2]{1}
Ordne den Gleichungen jeweils die entsprechende Definitionsmenge D (A-F) zu.

\zuordnen{
			R1={$\sqrt{x-23} = 5$},				% Response 1
			R2={$\sqrt{2x - 14} = 6$},			% Response 2
			R3={$\sqrt{56 - 2x} = 28$},				% Response 3
			R4={$\sqrt{x+14} = 0$},			% Response 4
			%% Moegliche Zuordnungen: %%
			A={$\{ x\in \mathbb{R} \, | \, x \leq 28 \}$}, 			%Moeglichkeit A  
			B={$\{ x\in \mathbb{R} \, | \, x \geq 28 \}$}, 			%Moeglichkeit B  
			C={$\{ x\in \mathbb{R} \, | \, x \geq 7 \}$}, 			%Moeglichkeit C  
			D={$\{ x\in \mathbb{R} \, | \, x \geq 23 \}$}, 			%Moeglichkeit D  
			E={$\{ x\in \mathbb{R} \, | \, x \geq 14  \}$}, 			%Moeglichkeit E  
			F={$\{ x\in \mathbb{R} \, | \, x \geq -14 \}$}, 			%Moeglichkeit F  
%% LOESUNG: %%
			A1={D},				% 1. richtige Zuordnung
			A2={C},				% 2. richtige Zuordnung
			A3={A},				% 3. richtige Zuordnung
			A4={F},				% 4. richtige Zuordnung
				}


\end{beispiel}

\info{\fbox{\begin{minipage}{0.98\textwidth}
Titel: Definitonsmeng Wurzelgleichung\\
Grundkompetenz(en): AG 1.2\\
Aufgabenformat: Zuordnungsformat\\
Klasse: 5\\
Quelle: KarWit
\end{minipage}}}

\newpage

\smallskip\begin{minipage}{1\textwidth}
 \flagUK\ \vspace{-0.5cm}

\subsection{AG 1.2 - 21 - Lineare Gleichung (Matura Wintertermin 22/23)}\smallskip

\end{minipage}

\begin{beispiel}[AG 1.2]{1}
Gegeben ist die folgende Gleichung in der Variablen $x\in\mathbb{Z}$:

$2\cdot x-c=0$ mit $c\in\mathbb{R}$

Gib alle reellen Zahlen $c$ an, für die diese Gleichung eine Lösung in $\mathbb{Z}$ hat.

\antwort{$\ldots, -4, -2, 0, 2, 4, \ldots$ (alle geraden ganzen Zahlen)}
\end{beispiel}

\info{\fbox{\begin{minipage}{0.98\textwidth}
Titel: Lineare Gleichung\\
Grundkompetenz(en): AG 1.2\\
Aufgabenformat: Offenes Antwortformat\\
Klasse: 5\\
Quelle: Matura Wintertermin 22/23
\end{minipage}}}

\newpage

\smallskip\begin{minipage}{1\textwidth}
\subsection{AG 1.2 - 22 - Äquivalente Aussagen}\smallskip

\end{minipage}

\begin{beispiel}[AG 1.2]{1}
Gegeben sind verschiedene Aussagen über Gleichungen in einer Unbekannten $x\in\mathbb{R}$ der Form: Gleichung A $\Leftrightarrow$ Gleichung B. Eine solche Aussage ist genau dann richtig, wenn Gleichung A äquivalent zu Gleichung B ist. Kreuze die beiden korrekten Aussagen an.\\

\multiplechoice[5]{  %Anzahl der Antwortmoeglichkeiten, Standard: 5
				L1={$\dfrac{x^3-2x}{5x^2}=0$\,$\Leftrightarrow$\,$x^3-2x=5x^2$},   %1. Antwortmoeglichkeit 
				L2={$4x^2+8x^4-12x=0$\,$\Leftrightarrow$\,$2x\cdot(2x+4x^3-6)=0$},   %2. Antwortmoeglichkeit
				L3={$3x^3-2x^5+x^6=1$\,$\Leftrightarrow$\,$4x^3-2x^5+x^6=x^3$},   %3. Antwortmoeglichkeit
				L4={$1-(x^4-3x^2+x^6)=12$\,$\Leftrightarrow$\,$1+x^4+3x^2-x^6=12$},   %4. Antwortmoeglichkeit
				L5={$(x-3)^2=5$\,$\Leftrightarrow$\,$x^2-6x=-4$},	 %5. Antwortmoeglichkeit
				L6={},	 %6. Antwortmoeglichkeit
				L7={},	 %7. Antwortmoeglichkeit
				L8={},	 %8. Antwortmoeglichkeit
				L9={},	 %9. Antwortmoeglichkeit
				%% LOESUNG: %%
				A1=2,  % 1. Antwort
				A2=5,	 % 2. Antwort
				A3=0,  % 3. Antwort
				A4=0,  % 4. Antwort
				A5=0,  % 5. Antwort
				}
\end{beispiel}

\info{\fbox{\begin{minipage}{0.98\textwidth}
Titel: Äquivalente Aussagen\\
Grundkompetenz(en): AG 1.2\\
Aufgabenformat: Mutiple Choice\\
Klasse: 5\\
Quelle: LaMA
\end{minipage}}}

\newpage

\smallskip\begin{minipage}{1\textwidth}
\subsection{AG 1.2 - 23 - Doppelbrüche}\smallskip

\end{minipage}

\begin{beispiel}[AG 1.2]{1}
Kreuze die beiden richtigen Umformungen an!

\multiplechoice[5]{  %Anzahl der Antwortmoeglichkeiten, Standard: 5
				L1={$\dfrac{\,\frac{2}{a}\,}{\frac{a}{1}}=\dfrac{2}{a\cdot a}$},   %1. Antwortmoeglichkeit 
				L2={$\dfrac{\,2\,}{\frac{2}{a}}=\dfrac{4}{a}$},   %2. Antwortmoeglichkeit
				L3={$\dfrac{\frac{a-2}{2}}{\frac{a}{2}}=\dfrac{a-2}{a}$},   %3. Antwortmoeglichkeit
				L4={$\dfrac{\frac{2a}{1}}{\frac{a}{2}}=2$},   %4. Antwortmoeglichkeit
				L5={$\dfrac{\frac{a-2}{2}}{\frac{2a}{a}}=a-2$},	 %5. Antwortmoeglichkeit
				L6={},	 %6. Antwortmoeglichkeit
				L7={},	 %7. Antwortmoeglichkeit
				L8={},	 %8. Antwortmoeglichkeit
				L9={},	 %9. Antwortmoeglichkeit
				%% LOESUNG: %%
				A1=1,  % 1. Antwort
				A2=3,	 % 2. Antwort
				A3=0,  % 3. Antwort
				A4=0,  % 4. Antwort
				A5=0,  % 5. Antwort
				}
\end{beispiel}

\info{\fbox{\begin{minipage}{0.98\textwidth}
Titel: Doppelbrüche\\
Grundkompetenz(en): AG 1.2\\
Aufgabenformat: Mutiple Choice\\
Klasse: 5\\
Quelle: LaMA
\end{minipage}}}

\newpage

\smallskip\begin{minipage}{1\textwidth}
\subsection{AG 1.2 - 24 - Äquivalente Gleichung mit zwei Variablen}\smallskip

\end{minipage}

\begin{beispiel}[AG 1.2]{1}
Gegeben ist die Gleichung $\dfrac{2a-b}{b}+1=b$ in $a,b\in\mathbb{R}^+$.

Kreuze die beiden nachstehenden Gleichungen an, die zur gegebenen Gleichung äquivalent sind!

\multiplechoice[5]{  %Anzahl der Antwortmoeglichkeiten, Standard: 5
				L1={$2a=b^2$},   %1. Antwortmoeglichkeit 
				L2={$\dfrac{2a-b}{b}=b+1$},   %2. Antwortmoeglichkeit
				L3={$2a-b+1=b^2$},   %3. Antwortmoeglichkeit
				L4={$\dfrac{2a-b}{b}-b=0$},   %4. Antwortmoeglichkeit
				L5={$\dfrac{2a}{b}=b$},	 %5. Antwortmoeglichkeit
				L6={},	 %6. Antwortmoeglichkeit
				L7={},	 %7. Antwortmoeglichkeit
				L8={},	 %8. Antwortmoeglichkeit
				L9={},	 %9. Antwortmoeglichkeit
				%% LOESUNG: %%
				A1=1,  % 1. Antwort
				A2=5,	 % 2. Antwort
				A3=0,  % 3. Antwort
				A4=0,  % 4. Antwort
				A5=0,  % 5. Antwort
				}
\end{beispiel}

\info{\fbox{\begin{minipage}{0.98\textwidth}
Titel: Äquivalente Gleichung mit zwei Variablen\\
Grundkompetenz(en): AG 1.2\\
Aufgabenformat: Mutiple Choice\\
Klasse: 5\\
Quelle: LaMA
\end{minipage}}}

\newpage

\smallskip\begin{minipage}{1\textwidth}
\subsection{AG 1.2 - i.19 - Gleichungen}\smallskip

\end{minipage}

\begin{beispiel}[AG 1.2]{1}
Ordne jeder Gleichung eine äquivalente Gleichung zu!

\zuordnen{
				R1={$\frac{1}{3}+x=\frac{2}{3}$},				% Response 1
				R2={$\frac{x}{4}=\frac{3}{2}$},				% Response 2
				R3={$x^2=36$},				% Response 3
				R4={$x^2-x=0$},				% Response 4
				%% Moegliche Zuordnungen: %%
				A={$\frac{x^2}{4}=9$}, 				%Moeglichkeit A  
				B={$x-1=0$}, 				%Moeglichkeit B  
				C={$x=6$}, 				%Moeglichkeit C  
				D={$1+x=2$}, 				%Moeglichkeit D  
				E={$x\cdot (x-1)=0$}, 				%Moeglichkeit E  
				F={$x=\frac{1}{3}$}, 				%Moeglichkeit F  
				%% LOESUNG: %%
				A1={F},				% 1. richtige Zuordnung
				A2={C},				% 2. richtige Zuordnung
				A3={A},				% 3. richtige Zuordnung
				A4={E},				% 4. richtige Zuordnung
				}
\end{beispiel}

\info{\fbox{\begin{minipage}{0.98\textwidth}
Titel: Gleichungen\\
Grundkompetenz(en): AG 1.2\\
Aufgabenformat: Zuordnungsformat\\
Klasse: 5\\
Quelle: Thema Mathematik Schularbeiten 5. Klasse
\end{minipage}}}

\newpage

\smallskip\begin{minipage}{1\textwidth}
\subsection{AG 1.2 - i.22 - Terme}\smallskip

\end{minipage}

\begin{beispiel}[AG 1.2]{1}
Schreibe den Term $8x - 5x \cdot (x - 3)$ als Summe.

\antwort{$8x - 5x^2 + 15x$ oder $23x - 5x^2$}
\end{beispiel}

\info{\fbox{\begin{minipage}{0.98\textwidth}
Titel: Terme\\
Grundkompetenz(en): AG 1.2\\
Aufgabenformat: Offenes Antwortformat\\
Klasse: 5\\
Quelle: Thema Mathematik Schularbeiten 5. Klasse
\end{minipage}}}

\newpage

\smallskip\begin{minipage}{1\textwidth}
\subsection{AG 1.2 - i.23 - Äquivalente Terme}\smallskip

\end{minipage}

\begin{beispiel}[AG 1.2]{1}
Gegeben sind vier Terme.

Ordne jedem Term in der linken Tabelle den passenden äquivalenten Term aus der rechten Tabelle zu!

\zuordnen{
				R1={$\frac{x-1}{x} - 2$},				% Response 1
				R2={$\frac{1}{x} \cdot (1-x)$},				% Response 2
				R3={$\frac{1}{x} \cdot (x+1)$},				% Response 3
				R4={$\frac{x+1}{x} - 1$},				% Response 4
				%% Moegliche Zuordnungen: %%
				A={$\frac{1}{x} - 1$}, 				%Moeglichkeit A  
				B={$\frac{1}{x}$}, 				%Moeglichkeit B  
				C={$-\frac{x+1}{x}$}, 				%Moeglichkeit C  
				D={$\frac{1}{x+1}$}, 				%Moeglichkeit D  
				E={$1-x$}, 				%Moeglichkeit E  
				F={$\frac{x+1}{x}$}, 				%Moeglichkeit F  
				%% LOESUNG: %%
				A1={C},				% 1. richtige Zuordnung
				A2={A},				% 2. richtige Zuordnung
				A3={F},				% 3. richtige Zuordnung
				A4={B},				% 4. richtige Zuordnung
				}
\end{beispiel}

\info{\fbox{\begin{minipage}{0.98\textwidth}
Titel: Äquivalente Terme\\
Grundkompetenz(en): AG 1.2\\
Aufgabenformat: Zuordnungsformat\\
Klasse: 5\\
Quelle: Mathematik verstehen, Matura, Malle
\end{minipage}}}

\newpage

\smallskip\begin{minipage}{1\textwidth}
\subsection{AG 1.2 - i.28 - Äquivalente Gleichungen}\smallskip

\end{minipage}

\begin{beispiel}[AG 1.2]{1}
Gegeben ist die Gleichung $\frac{a \cdot (b-c)}{d}=b-a$.
				
				Kreuze die Gleichungen an, die zu dieser Gleichung äquivalent sind!
				
				\multiplechoice[5]{  %Anzahl der Antwortmoeglichkeiten, Standard: 5
								L1={$a=\frac{bd}{b-c+d}$},   %1. Antwortmoeglichkeit 
								L2={$b=a \cdot \frac{a-d}{c-d}$},   %2. Antwortmoeglichkeit
								L3={$b=a \cdot \frac{c-d}{a-d}$},   %3. Antwortmoeglichkeit
								L4={$c=b+d-\frac{bd}{a}$},   %4. Antwortmoeglichkeit
								L5={$c=b+\frac{d(a-b)}{a}$},	 %5. Antwortmoeglichkeit
								L6={},	 %6. Antwortmoeglichkeit
								L7={},	 %7. Antwortmoeglichkeit
								L8={},	 %8. Antwortmoeglichkeit
								L9={},	 %9. Antwortmoeglichkeit
								%% LOESUNG: %%
								A1=1,  % 1. Antwort
								A2=3,	 % 2. Antwort
								A3=4,  % 3. Antwort
								A4=5,  % 4. Antwort
								A5=0,  % 5. Antwort
								}
\end{beispiel}

\info{\fbox{\begin{minipage}{0.98\textwidth}
Titel: Äquivalente Gleichungen\\
Grundkompetenz(en): AG 1.2\\
Aufgabenformat: Mutiple Choice\\
Klasse: 5\\
Quelle: Mathematik verstehen, Matura, Malle
\end{minipage}}}

\newpage

\smallskip\begin{minipage}{1\textwidth}
{\color{green!40!black!60!}{\fbox{Anzahl weiterer Variationen dieser Aufgabe: 1}}}\vspace{-0.5cm}

\subsection{AG 1.2 - i.29 - Definitionsmenge}\smallskip

\end{minipage}

\begin{beispiel}[AG 1.2]{1}
Ordne jedem Term in der linken Tabelle die Definitionsmenge aus der rechten Tabelle zu! 
	
	\zuordnen{
					R1={$\sqrt{x-1}$},				% Response 1
					R2={$(x-1)^2$},				% Response 2
					R3={$\frac{1}{x}+1$},				% Response 3
					R4={$\frac{x}{x^2-4}$},				% Response 4
					%% Moegliche Zuordnungen: %%
					A={$\mathbb{R}$}, 				%Moeglichkeit A  
					B={$\mathbb{R}^+ \cup \mathbb{R}^- $}, 				%Moeglichkeit B  
					C={$\mathbb{R} \backslash \{1\}$}, 				%Moeglichkeit C  
					D={$\mathbb{R} \backslash \{-2;2\}$}, 				%Moeglichkeit D  
					E={$[1;\infty$)}, 				%Moeglichkeit E  
					F={$\mathbb{R} \backslash \{\pm \sqrt{2}\}$}, 				%Moeglichkeit F  
					%% LOESUNG: %%
					A1={E},				% 1. richtige Zuordnung
					A2={A},				% 2. richtige Zuordnung
					A3={B},				% 3. richtige Zuordnung
					A4={D},				% 4. richtige Zuordnung
					}
\end{beispiel}

\info{\fbox{\begin{minipage}{0.98\textwidth}
Titel: Definitionsmenge\\
Grundkompetenz(en): AG 1.2\\
Aufgabenformat: Zuordnungsformat\\
Klasse: 5\\
Quelle: Mathematik verstehen 5
\end{minipage}}}

\newpage

\smallskip\begin{minipage}{1\textwidth}
\subsection{AG 1.2 - i.31 - Nichtraucher}\smallskip

\end{minipage}

\begin{beispiel}[AG 1.2]{1}
Eine Firma hat B Beschäftigte. Der Anteil der männlichen Angestellten liegt bei 70\,\%.\\
40\,\% aller männlichen Mitarbeiter dieses Betriebes rauchen.

Beschreibe die Anzahl $N$ der männlichen Nichtraucher dieser Firma mit Hilfe eines geeigneten Terms!\leer

$N=$\,\antwort[\rule{3cm}{0.3pt}]{$B\cdot 0,7\cdot 0,6=B\cdot 0,42$}
\end{beispiel}

\info{\fbox{\begin{minipage}{0.98\textwidth}
Titel: Nichtraucher\\
Grundkompetenz(en): AG 1.2\\
Aufgabenformat: Offenes Antwortformat\\
Klasse: 5\\
Quelle: thema mathematik 8 Schularbeiten
\end{minipage}}}

\newpage

\smallskip\begin{minipage}{1\textwidth}
\subsection{AG 1.2 - i.32 - Äquivalenz von Termen}\smallskip

\end{minipage}

\begin{beispiel}[AG 1.2]{1}
Die nachfolgende Aufgabenstellung bezieht sich auf die Äquivalenz von Termen.

Ordne den vier gegebenen Termen jeweils den dazu äquivalenten Term zu.

\zuordnen{
				R1={$x^2-9$},				% Response 1
				R2={$x^2+6x+9$},				% Response 2
				R3={$x^2-3x$},				% Response 3
				R4={$x^2+3x$},				% Response 4
				%% Moegliche Zuordnungen: %%
				A={$(x-3)^2$}, 				%Moeglichkeit A  
				B={$(x-3)\cdot(x+3)$}, 				%Moeglichkeit B  
				C={$3\cdot(x+1)\cdot x$}, 				%Moeglichkeit C  
				D={$(x+3)^2$}, 				%Moeglichkeit D  
				E={$x\cdot(x+2)+x$}, 				%Moeglichkeit E  
				F={$x\cdot(x-3)$}, 				%Moeglichkeit F  
				%% LOESUNG: %%
				A1={B},				% 1. richtige Zuordnung
				A2={D},				% 2. richtige Zuordnung
				A3={F},				% 3. richtige Zuordnung
				A4={E},				% 4. richtige Zuordnung
				}
\end{beispiel}

\info{\fbox{\begin{minipage}{0.98\textwidth}
Titel: Äquivalenz von Termen\\
Grundkompetenz(en): AG 1.2\\
Aufgabenformat: Zuordnungsformat\\
Klasse: 5\\
Quelle: Dimensionen Mathematik 6 Schularbeiten-Trainer
\end{minipage}}}

\newpage

\smallskip\begin{minipage}{1\textwidth}
\subsection{AG 1.2 - i.33 - Äquivalenz von Termen}\smallskip

\end{minipage}

\begin{beispiel}[AG 1.2]{1}
Die nachfolgende Aufgabenstellung bezieht sich auf die Äquivalenz von Termen.

Ordne den vier gegebenen Termen jeweils den dazu äquivalenten Term zu.

\zuordnen{
				R1={$\frac{x}{2}+\left(-\dfrac{x}{3}\right)$},				% Response 1
				R2={$\left(-\frac{x}{2}\right)\cdot\left(-\frac{x}{3}\right)$},				% Response 2
				R3={$\left(-\frac{x}{2}\right):\frac{x}{3}$},				% Response 3
				R4={$\left(-\frac{x}{2}\right)-\left(+\frac{x}{3}\right)$},				% Response 4
				%% Moegliche Zuordnungen: %%
				A={$\frac{5}{6}$}, 				%Moeglichkeit A  
				B={$-\frac{3}{2}$}, 				%Moeglichkeit B  
				C={$-\frac{5x}{6}$}, 				%Moeglichkeit C  
				D={$\frac{x}{6}$}, 				%Moeglichkeit D  
				E={$-\frac{x^2}{6}$}, 				%Moeglichkeit E  
				F={$\frac{x^2}{6}$}, 				%Moeglichkeit F  
				%% LOESUNG: %%
				A1={D},				% 1. richtige Zuordnung
				A2={F},				% 2. richtige Zuordnung
				A3={B},				% 3. richtige Zuordnung
				A4={C},				% 4. richtige Zuordnung
				}
\end{beispiel}

\info{\fbox{\begin{minipage}{0.98\textwidth}
Titel: Äquivalenz von Termen\\
Grundkompetenz(en): AG 1.2\\
Aufgabenformat: Zuordnungsformat\\
Klasse: 5\\
Quelle: Dimensionen Mathematik 6 Schularbeiten-Trainer
\end{minipage}}}

\newpage

\smallskip\begin{minipage}{1\textwidth}
\subsection{AG 1.2 - i.34 - Umformen}\smallskip

\end{minipage}

\begin{beispiel}[AG 1.2]{1}
Gegeben ist die nachfolgende Formel: $F=\frac{2a}{3}-c\cdot\frac{a-1}{b}$.

Forme die Formel nach $a$ um.\leer

$a=\,\antwort[\rule{3cm}{0.3pt}]{\frac{3bF-3c}{2b-3c}}$
\end{beispiel}

\info{\fbox{\begin{minipage}{0.98\textwidth}
Titel: Umformen\\
Grundkompetenz(en): AG 1.2\\
Aufgabenformat: Offenes Antwortformat\\
Klasse: 5\\
Quelle: Dimensionen Mathematik 6 Schularbeiten-Trainer
\end{minipage}}}

\newpage

\smallskip\begin{minipage}{1\textwidth}
{\color{green!40!black!60!}{\fbox{Anzahl weiterer Variationen dieser Aufgabe: 1}}}\vspace{-0.5cm}

\subsection{AG 1.2 - i.35 - Umformen von Termen}\smallskip

\end{minipage}

\begin{beispiel}[AG 1.2]{1}
Gegeben ist der nachfolgende Term: $\frac{3}{x^2-1}-\frac{2}{x^2-x}$.

Der Term kann in der Form $\frac{a\cdot x+b}{x^3+c\cdot x}$ geschrieben werden. Gib die Werte für $a,b$ und $c$ an.\leer

$a=\,\antwort[\rule{3cm}{0.3pt}]{1}$\leer

$b=\,\antwort[\rule{3cm}{0.3pt}]{-2}$\leer

$c=\,\antwort[\rule{3cm}{0.3pt}]{-1}$
\end{beispiel}

\info{\fbox{\begin{minipage}{0.98\textwidth}
Titel: Umformen von Termen\\
Grundkompetenz(en): AG 1.2\\
Aufgabenformat: Offenes Antwortformat\\
Klasse: 6\\
Quelle: Dimensionen Mathematik 6 Schularbeiten-Trainer
\end{minipage}}}

\newpage

\smallskip\begin{minipage}{1\textwidth}
{\color{green!40!black!60!}{\fbox{Anzahl weiterer Variationen dieser Aufgabe: 3}}}\vspace{-0.5cm}

\subsection{AG 1.2 - i.38 - Falsches Gleichungslösen}\smallskip

\end{minipage}

\begin{beispiel}[AG 1.2]{1}
Eine Person löst die Gleichung $x^2-9=2x-6$ bezüglich der Grundmenge $G=\mathbb{R}$ wie folgt:
\begin{align}
(x+3)\cdot(x-3)&=2\cdot(x-3)\qquad |:(x-3)\\
x+3&=2\qquad |-3\\
x&=-1\\
L&=\{-1\}\\
\end{align}

Begründe, warum die Lösung nicht korrekt ist, und gib die korrekte Lösungsmenge an.

\antwort{Dividiert man die Gleichung durch den Term $x-3$, so geht die Lösung $x_2=3$ verloren.\\
Für $x=3$ würde die Division durch $x-3$ einer Division durch 0 entsprechen.

$L\{-1;3\}$}
\end{beispiel}

\info{\fbox{\begin{minipage}{0.98\textwidth}
Titel: Falsches Gleichungslösen\\
Grundkompetenz(en): AG 1.2\\
Aufgabenformat: Offenes Antwortformat\\
Klasse: 5\\
Quelle: Dimensionen Mathematik 6 Schularbeiten-Trainer
\end{minipage}}}

\newpage

\smallskip\begin{minipage}{1\textwidth}
\subsection{AG 1.2 - i.40 - Äquivalenzumformung}\smallskip

\end{minipage}

\begin{beispiel}[AG 1.2]{1}
Nicht jede Umformung, die man auf beiden Seiten einer Gleichung anwendet, erfüllt die an eine Äquivalenzumformung gestellten Bedingungen.

Erläutere an einem konkreten Beispiel, warum das Quadrieren der beiden Seiten einer Gleichung eventuell keine Äquivalenzumformung darstellt.

\antwort{
Die Gleichung $x=3$ hat die Lösungsmenge $L=\{3\}$\\
Die Gleichung $x^2=9$ hat hingegen die Lösungsmenge $L=\{ \pm 3\}$
}
\end{beispiel}

\info{\fbox{\begin{minipage}{0.98\textwidth}
Titel: Äquivalenzumformung\\
Grundkompetenz(en): AG 1.2\\
Aufgabenformat: Offenes Antwortformat\\
Klasse: 5\\
Quelle: Dimensionen 5 Schularbeitentrainer
\end{minipage}}}

\newpage

\smallskip\begin{minipage}{1\textwidth}
\subsection{AG 1.2 - i.41 - Äquivalenzumformung}\smallskip

\end{minipage}

\begin{beispiel}[AG 1.2]{1}
Gegeben ist die Gleichung $x\cdot (x-2)=0$ in der Grundmenge $G=\mathbb{R}$.

Erläutere, warum die nachstehend angeführte Umformung keine Äquivalenzumformung darstellt.
\begin{align*}
x\cdot (x-2)=0 \qquad |:x
\end{align*}

\antwort{
Die Gleichung $x\cdot (x-2)=0$ hat die Lösungsmenge $L=\{0, 2\}$\\
Die Gleichung $(x-2)=0$ hat die Lösungsmenge $L=\{2\}$.\\ 
Die Lösung 0 geht somit verloren.
}
\end{beispiel}

\info{\fbox{\begin{minipage}{0.98\textwidth}
Titel: Äquivalenzumformung\\
Grundkompetenz(en): AG 1.2\\
Aufgabenformat: Offenes Antwortformat\\
Klasse: 5\\
Quelle: Dimensionen 5 Schularbeitentrainer
\end{minipage}}}

\newpage

\smallskip\begin{minipage}{1\textwidth}
\subsection{AG 1.2 - i.45 - Formel, Gleichung, Variable}\smallskip

\end{minipage}

\begin{beispiel}[AG 1.2]{1}
Für den von einem Fahrzeug bei einer konstanten Geschwindigkeit von $v$\,km/h in $t$ Stunden zurückgelegten Weg $s$ (in km) gilt: $s=v\cdot t$.

\lueckentext[-0.2]{
				text={Beim Ausdruck $s=v\cdot t$ handelt es sich um eine \gap, weil \gap .}, 	%Lueckentext Luecke=\gap
				L1={Variable}, 		%1.Moeglichkeit links  
				L2={Gleichung}, 		%2.Moeglichkeit links
				L3={Formel}, 		%3.Moeglichkeit links
				R1={die vorkommenden Größen unterschiedliche Werte annehmen können}, 		%1.Moeglichkeit rechts 
				R2={im Ausdruck ein "`="' vorkommt}, 		%2.Moeglichkeit rechts
				R3={der Ausdruck eine allgemeingültige Beziehung zwischen Größen beschreibt}, 		%3.Moeglichkeit rechts
				%% LOESUNG: %%
				A1=3,   % Antwort links
				A2=3		% Antwort rechts 
				}
\end{beispiel}

\info{\fbox{\begin{minipage}{0.98\textwidth}
Titel: Formel, Gleichung, Variable\\
Grundkompetenz(en): AG 1.2\\
Aufgabenformat: Lückentext\\
Klasse: 5\\
Quelle: Dimensionen 5 Schularbeitentrainer
\end{minipage}}}

\newpage

\smallskip\begin{minipage}{1\textwidth}
\subsection{AG 1.2 - i.46 - Gleichung, Term, Formel}\smallskip

\end{minipage}

\begin{beispiel}[AG 1.2]{1}
Wie in der Abbildung dargestellt, wird von einem Rechteck rechts unten ein rechteckiger Teil entfernt. Längenangaben in dm.
\begin{center}
\psset{xunit=0.5cm,yunit=0.5cm,algebraic=true,dimen=middle,dotstyle=o,dotsize=5pt 0,linewidth=1.6pt,arrowsize=3pt 2,arrowinset=0.25}
\begin{pspicture*}(-0.5044024890232,-0.9486080923277985)(11.110703311184466,7.3683259227047335)
\pspolygon[linewidth=2pt,fillcolor=black,fillstyle=solid,opacity=0.1](0,0)(7,0)(7,4)(10,4)(10,6)(0,6)
\psline[linewidth=2pt](0,0)(7,0)
\psline[linewidth=2pt](7,0)(7,4)
\psline[linewidth=2pt](7,4)(10,4)
\psline[linewidth=2pt](10,4)(10,6)
\psline[linewidth=2pt](10,6)(0,6)
\psline[linewidth=2pt](0,6)(0,0)
\rput[bl](3.516213197730779,-0.8068751440437976){$a$}
\rput[bl](7.502829505379414,2.009267935373783){4}
\rput[bl](10.50913032754068,5.015568757535048){$b$}
\rput[bl](5.019363608811411,6.616750717164417){10}
\end{pspicture*}
\end{center}
\lueckentext{
				text={Mithilfe der Variablen \gap kann man \gap\\ $10\cdot (b+4)-4\cdot (10-a)$ für die Berechnung des Flächeninhalts der dargestellten\\ Figur angeben.}, 	%Lueckentext Luecke=\gap
				L1={4 und 10}, 		%1.Moeglichkeit links  
				L2={$(10-a)$ und $(b+4)$}, 		%2.Moeglichkeit links
				L3={$a$ und $b$}, 		%3.Moeglichkeit links
				R1={die Gleichung}, 		%1.Moeglichkeit rechts 
				R2={den Term}, 		%2.Moeglichkeit rechts
				R3={die Formel}, 		%3.Moeglichkeit rechts
				%% LOESUNG: %%
				A1=3,   % Antwort links
				A2=2		% Antwort rechts 
				}
\end{beispiel}

\info{\fbox{\begin{minipage}{0.98\textwidth}
Titel: Gleichung, Term, Formel\\
Grundkompetenz(en): AG 1.2\\
Aufgabenformat: Lückentext\\
Klasse: 5\\
Quelle: Dimensionen 5 Schularbeitentrainer
\end{minipage}}}

\newpage

\smallskip\begin{minipage}{1\textwidth}
\subsection{AG 1.2 - i.47 - Unlösbar}\smallskip

\end{minipage}

\begin{beispiel}[AG 1.2]{1}
Gegeben ist die Gleichung $a\cdot x=b$  mit $a$, $b\in \mathbb{R}$ in der Variablen $x$ und der Grundmenge $G=\mathbb{R}$.

Gib an, welche Bedingungen die beiden Parameter $a$ und/bzw. $b$ erfüllen müssen, damit die Gleichung nicht lösbar ist.

\antwort{Für $a=0$ und $b\neq0$ ist die Gleichung nicht lösbar.}
\end{beispiel}

\info{\fbox{\begin{minipage}{0.98\textwidth}
Titel: Unlösbar\\
Grundkompetenz(en): AG 1.2\\
Aufgabenformat: Offenes Antwortformat\\
Klasse: 5\\
Quelle: Dimensionen 5 Schularbeitentrainer
\end{minipage}}}

\newpage

\smallskip\begin{minipage}{1\textwidth}
\subsection{AG 1.2 - i.48 - Lösungen}\smallskip

\end{minipage}

\begin{beispiel}[AG 1.2]{1}
Gegeben ist die nachstehende Gleichung in der Variablen $x$. Die Grundmenge der Gleichung wird mit $G$ bezeichnet. 
\begin{align*}
a\cdot x \cdot (x+2)=0 \qquad (a\in\mathbb{R})
\end{align*}

Kreuze die zutreffenden Aussagen an.

\multiplechoice[5]{  %Anzahl der Antwortmoeglichkeiten, Standard: 5
				L1={Die Gleichung hat unabhängig von $a$ und $G$ immer genau eine Lösung.},   %1. Antwortmoeglichkeit 
				L2={Bei bestimmter Wahl von $a$ kann die Gleichung mehr als zwei Lösungen besitzen.},   %2. Antwortmoeglichkeit
				L3={Es ist möglich, dass die Gleichung bei bestimmter Wahl von $G$ genau eine Lösung besitzt.},   %3. Antwortmoeglichkeit
				L4={Für $a\neq 0$ besitzt die Gleichung unabhängig von $G$ immer genau zwei Lösungen.},   %4. Antwortmoeglichkeit
				L5={Gilt $G=\mathbb{R}$, kann $a$ so gewählt werden, dass die Gleichung genau eine Lösung besitzt.},	 %5. Antwortmoeglichkeit
				L6={},	 %6. Antwortmoeglichkeit
				L7={},	 %7. Antwortmoeglichkeit
				L8={},	 %8. Antwortmoeglichkeit
				L9={},	 %9. Antwortmoeglichkeit
				%% LOESUNG: %%
				A1=2,  % 1. Antwort
				A2=3,	 % 2. Antwort
				A3=0,  % 3. Antwort
				A4=0,  % 4. Antwort
				A5=0,  % 5. Antwort
				}
\end{beispiel}

\info{\fbox{\begin{minipage}{0.98\textwidth}
Titel: Lösungen\\
Grundkompetenz(en): AG 1.2\\
Aufgabenformat: Mutiple Choice\\
Klasse: 5\\
Quelle: Dimensionen 5 Schularbeitentrainer
\end{minipage}}}

\newpage

\smallskip\begin{minipage}{1\textwidth}
\subsection{AG 1.2 - i.49 - Lösungen}\smallskip

\end{minipage}

\begin{beispiel}[AG 1.2]{1}
Gegeben sind vier Aussagen über die Lösungsmengen von Gleichungen. Die Grundmenge der Gleichungen ist $\mathbb{R}$.

\zuordnen{
				R1={$L=\mathbb{R}$},				% Response 1
				R2={$L=\{\}$},				% Response 2
				R3={$L=\{2\}$},				% Response 3
				R4={$L=\{0,\, 2\}$},				% Response 4
				%% Moegliche Zuordnungen: %%
				A={$x-2=0$}, 				%Moeglichkeit A  
				B={$0=x\cdot (x-2)$}, 				%Moeglichkeit B  
				C={$x\cdot (x-2)=-2\cdot x+x^2$}, 				%Moeglichkeit C  
				D={$x^2-4=0$}, 				%Moeglichkeit D  
				E={$3x+1=4$}, 				%Moeglichkeit E  
				F={$x^2+1=0$}, 				%Moeglichkeit F  
				%% LOESUNG: %%
				A1={C},				% 1. richtige Zuordnung
				A2={F},				% 2. richtige Zuordnung
				A3={A},				% 3. richtige Zuordnung
				A4={B},				% 4. richtige Zuordnung
				}
\end{beispiel}

\info{\fbox{\begin{minipage}{0.98\textwidth}
Titel: Lösungen\\
Grundkompetenz(en): AG 1.2\\
Aufgabenformat: Zuordnungsformat\\
Klasse: 5\\
Quelle: Dimensionen 5 Schularbeitentrainer
\end{minipage}}}

\newpage

\smallskip\begin{minipage}{1\textwidth}
\subsection{AG 1.2 - i.50 - Rechengesetze}\smallskip

\end{minipage}

\begin{beispiel}[AG 1.2]{1}
Nachstehend werden einige Rechengesetze sowie mehrere äquivalente Gleichungen angeführt.

Ordne den Rechengesetzen jeweils jene Gleichung zu, die das betreffende Rechengesetz zum Ausdruck bringt.

\zuordnen{
				R1={Kommutativgesetz für die Multiplikation},				% Response 1
				R2={Kommutativgesetz für die Addition},				% Response 2
				R3={Assoziativgesetz für die Multiplikation},				% Response 3
				R4={Distributivgesetz},				% Response 4
				%% Moegliche Zuordnungen: %%
				A={$(x+3)+2=x+5$}, 				%Moeglichkeit A  
				B={$2-(x-5)=7-x$}, 				%Moeglichkeit B  
				C={$0,5\cdot (4x)=2x$}, 				%Moeglichkeit C  
				D={$1+2x=2x+1$}, 				%Moeglichkeit D  
				E={$2\cdot (x+b)=2x+2b$}, 				%Moeglichkeit E  
				F={$x\cdot (2+5)=7x$}, 				%Moeglichkeit F  
				%% LOESUNG: %%
				A1={F},				% 1. richtige Zuordnung
				A2={D},				% 2. richtige Zuordnung
				A3={C},				% 3. richtige Zuordnung
				A4={E},				% 4. richtige Zuordnung
				}
\end{beispiel}

\info{\fbox{\begin{minipage}{0.98\textwidth}
Titel: Rechengesetze\\
Grundkompetenz(en): AG 1.2\\
Aufgabenformat: Zuordnungsformat\\
Klasse: 5\\
Quelle: Dimensionen 5 Schularbeitentrainer
\end{minipage}}}

\newpage

\smallskip\begin{minipage}{1\textwidth}
\subsection{AG 1.2 - i.51 - Rechengesetze}\smallskip

\end{minipage}

\begin{beispiel}[AG 1.2]{1}
Die Gleichung $0,5x=x-2$ wird mithilfe von Äquivalenzumformungen wie folgt gelöst:

\underline{Schritt 1}: Beide Seiten der Gleichung werden mit der Zahl 2 multipliziert\\ 
Ergebnis: $x=2x-4$\\
\underline{Schritt 2}: Zu beiden Seiten der Gleichung wird $(4-x)$ addiert.\\ 
Ergebnis: $4=x$

\lueckentext{
				text={Beim Schritt 1 kommt (u.a.) das \gap , beim Schritt 2 kommt (u.a.) das \gap zur Anwendung.}, 	%Lueckentext Luecke=\gap
				L1={Kommutativgesetz für die Addition}, 		%1.Moeglichkeit links  
				L2={Assoziativgesetz für die Addition}, 		%2.Moeglichkeit links
				L3={Distributivgesetz}, 		%3.Moeglichkeit links
				R1={Kommutativgesetz für die Multiplikation}, 		%1.Moeglichkeit rechts 
				R2={Assoziativgesetz für die Addition}, 		%2.Moeglichkeit rechts
				R3={Distributivgesetz}, 		%3.Moeglichkeit rechts
				%% LOESUNG: %%
				A1=3,   % Antwort links
				A2=2		% Antwort rechts 
				}
\end{beispiel}

\info{\fbox{\begin{minipage}{0.98\textwidth}
Titel: Rechengesetze\\
Grundkompetenz(en): AG 1.2\\
Aufgabenformat: Lückentext\\
Klasse: 5\\
Quelle: Dimensionen 5 Schularbeitentrainer
\end{minipage}}}

\newpage

\smallskip\begin{minipage}{1\textwidth}
{\color{green!40!black!60!}{\fbox{Anzahl weiterer Variationen dieser Aufgabe: 1}}}\vspace{-0.5cm}

\subsection{AG 1.2 - i.52 - Rechengesetze}\smallskip

\end{minipage}

\begin{beispiel}[AG 1.2]{1}
Die Gleichung $a\cdot (x+b)-a\cdot b=U$ wird schrittweise nach $x$ umgeformt. Dabei werden verschiedene Rechengesetze angewendet.

\underline{Schritt 1}: Beidseitge Addition des Produkts $a\cdot b$ (von rechts) führt zu $a\cdot (x+b)=U+a\cdot b$.\\ 
\underline{Schritt 2}: Ausmultiplizieren der Klammer und beidseitige Subtraktion von $a\cdot b$ führt zu $a\cdot x=U$.\\
\underline{Schritt 3}: Vertauschen der beiden Faktoren auf der linken Seite der Gleichung führt zu $x\cdot a=U$.\\
\underline{Schritt 4}: Beidseitige Multiplikation mit $\frac{1}{a}$ (von rechts) führt zum Ergebnis $x=\frac{U}{a}$.\\

Ordne den vier Schritten jeweils jenes Rechengesetz zu, das beim entsprechenden Schritt zur Anwendung gelangt.

\zuordnen[-0.1]{
				R1={Schritt 1},				% Response 1
				R2={Schritt 2},				% Response 2
				R3={Schritt 3},				% Response 3
				R4={Schritt 4},				% Response 4
				%% Moegliche Zuordnungen: %%
				A={Kommutativgesetz bezüglich der Addition}, 				%Moeglichkeit A  
				B={Kommutativgesetz bezüglich der Multiplikation}, 				%Moeglichkeit B  
				C={Assoziativgesetz bezüglich der Addition}, 				%Moeglichkeit C  
				D={Assoziativgesetz bezüglich der Multiplikation}, 				%Moeglichkeit D  
				E={Distributivgesetz der Addition bezüglich der Multiplikation}, 				%Moeglichkeit E  
				F={Distributivgesetz der Multiplikation bezüglich der Addition}, 				%Moeglichkeit F  
				%% LOESUNG: %%
				A1={C},				% 1. richtige Zuordnung
				A2={F},				% 2. richtige Zuordnung
				A3={B},				% 3. richtige Zuordnung
				A4={D},				% 4. richtige Zuordnung
				}
\end{beispiel}

\info{\fbox{\begin{minipage}{0.98\textwidth}
Titel: Rechengesetze\\
Grundkompetenz(en): AG 1.2\\
Aufgabenformat: Zuordnungsformat\\
Klasse: 5\\
Quelle: Dimensionen 5 Schularbeitentrainer
\end{minipage}}}

\newpage

\smallskip\begin{minipage}{1\textwidth}
\subsection{AG 1.2 - i.55 - Mathematische Begriffe}\smallskip

\end{minipage}

\begin{beispiel}[AG 1.2]{1}
Gegeben ist $60<4x$.

\lueckentext{
				text={Ändert man \gap das Rechenzeichen in ein =, dann ist \gap in $\mathbb{N}$ lösbar.}, 	%Lueckentext Luecke=\gap
				L1={in der Gleichung}, 		%1.Moeglichkeit links  
				L2={im Term}, 		%2.Moeglichkeit links
				L3={in der Ungleichung}, 		%3.Moeglichkeit links
				R1={die Gleichung}, 		%1.Moeglichkeit rechts 
				R2={der Term}, 		%2.Moeglichkeit rechts
				R3={die Ungleichung}, 		%3.Moeglichkeit rechts
				%% LOESUNG: %%
				A1=3,   % Antwort links
				A2=1		% Antwort rechts 
				}
\end{beispiel}

\info{\fbox{\begin{minipage}{0.98\textwidth}
Titel: Mathematische Begriffe\\
Grundkompetenz(en): AG 1.2\\
Aufgabenformat: Lückentext\\
Klasse: 5\\
Quelle: Lösungswege Maturatraining
\end{minipage}}}

\newpage


\end{document}


% Aufgabenliste: 4, 10