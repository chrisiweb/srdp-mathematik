

\documentclass[a4paper,12pt]{article}
\usepackage{geometry}
\geometry{a4paper,left=18mm,right=18mm, top=2cm, bottom=2cm}
 

\usepackage{lmodern}
\usepackage[T1]{fontenc}
\usepackage[utf8]{inputenc}
\usepackage[ngerman]{babel}
\usepackage[solution_on, random=0, info_off]{srdp-mathematik} % solution_on/off, random, info_on/off
%
%
%\pagestyle{plain} %PAGESTYLE: empty, plain
%\onehalfspacing %Zeilenabstand
%\setcounter{secnumdepth}{-1} % keine Nummerierung der Überschriften
%
%
%%%%%%%%%%%%%%%%%%%%%%%%%%%%%%%%%%%%%%%%%%%%%%%%%%%%%%%%%%%%%%%%%
%%%%%%%%%%%%%%%%%%%%%% DOKUMENT - ANFANG %%%%%%%%%%%%%%%%%%%%%%%%
%%%%%%%%%%%%%%%%%%%%%%%%%%%%%%%%%%%%%%%%%%%%%%%%%%%%%%%%%%%%%%%%%
%
%

%\NewEnviron{test}{
%	\begin{enumerate}
%		\item {\BODY}
%	\end{enumerate}
%}

\begin{document}
%\begin{test}
%Written text
%
%\begin{center}
%dasd
%\end{center}
%
%\end{test}

\begin{beispiel}{0} %PUNKTE DES BEISPIELS
dasdas

\lueckentext{
				text={das}, 	%Lueckentext Luecke=\gap
				L1={•}, 		%1.Moeglichkeit links  
				L2={•}, 		%2.Moeglichkeit links
				L3={•}, 		%3.Moeglichkeit links
				R1={•}, 		%1.Moeglichkeit rechts 
				R2={•}, 		%2.Moeglichkeit rechts
				R3={•}, 		%3.Moeglichkeit rechts
				%% LOESUNG: %%
				A1=0,   % Antwort links
				A2=0		% Antwort rechts 
				}
\end{beispiel}

\begin{beispiel}[AG 1.1]{0} %PUNKTE DES BEISPIELS
Das ist ein Test
\langmultiplechoice[5]{  %Anzahl der Antwortmoeglichkeiten, Standard: 5
				L1={$\Vek{1}{3}{4} = \Vek{4}{4}{1} \cdot \Vek{2}{1}{x}$},   %1. Antwortmoeglichkeit 
				L2={•},   %2. Antwortmoeglichkeit
				L3={•},   %3. Antwortmoeglichkeit
				L4={•},   %4. Antwortmoeglichkeit
				L5={•},	 %5. Antwortmoeglichkeit
				L6={},	 %6. Antwortmoeglichkeit
				L7={},	 %7. Antwortmoeglichkeit
				L8={},	 %8. Antwortmoeglichkeit
				L9={},	 %9. Antwortmoeglichkeit
				%% LOESUNG: %%
				A1=1,  % 1. Antwort
				A2=5,	 % 2. Antwort
				A3=0,  % 3. Antwort
				A4=0,  % 4. Antwort
				A5=0,  % 5. Antwort
				}
\end{beispiel}

\begin{beispiel}{0} %PUNKTE DES BEISPIELS
dsadsdas

\rfmultiplechoice[5]{Aussage}{  %Anzahl der Antwortmoeglichkeiten, Standard: 5
				L1={dasd},   %1. Antwortmoeglichkeit 
				L2={das},   %2. Antwortmoeglichkeit
				L3={das},   %3. Antwortmoeglichkeit
				L4={das},   %4. Antwortmoeglichkeit
				L5={•},	 %5. Antwortmoeglichkeit
				L6={},	 %6. Antwortmoeglichkeit
				L7={},	 %7. Antwortmoeglichkeit
				L8={},	 %8. Antwortmoeglichkeit
				L9={},	 %9. Antwortmoeglichkeit
				%% LOESUNG: %%
				A1=0,  % 1. Antwort
				A2=0,	 % 2. Antwort
				A3=0,  % 3. Antwort
				A4=0,  % 4. Antwort
				A5=0,  % 5. Antwort
				}
\end{beispiel}

\begin{beispiel}{0} %PUNKTE DES BEISPIELS
Das ist ein Test

\zuordnen{
				R1={daskdl},				% Response 1
				R2={daksöl},				% Response 2
				R3={dakslö},				% Response 3
				R4={daslö},				% Response 4
				%% Moegliche Zuordnungen: %%
				A={asdl}, 				%Moeglichkeit A  
				B={dlas}, 				%Moeglichkeit B  
				C={daslöd}, 				%Moeglichkeit C  
				D={daslä}, 				%Moeglichkeit D  
				E={daslä}, 				%Moeglichkeit E  
				F={daslä}, 				%Moeglichkeit F  
				%% LOESUNG: %%
				A1={•},				% 1. richtige Zuordnung
				A2={•},				% 2. richtige Zuordnung
				A3={•},				% 3. richtige Zuordnung
				A4={•},				% 4. richtige Zuordnung
				}


\end{beispiel}
\end{document}

% Aufgabenliste: 4, 10