\documentclass[a4paper,12pt]{article}

\usepackage{geometry}
\geometry{a4paper,left=18mm,right=18mm, top=2cm, bottom=2cm} 

\usepackage{lmodern}
\usepackage[T1]{fontenc}
\usepackage[utf8]{inputenc}
\usepackage[ngerman]{babel}
\usepackage{tabularray}
\usepackage[solution_on, random=0]{srdp-mathematik} % solution_on/off, random

\pagestyle{plain} %PAGESTYLE: empty, plain
%\onehalfspacing %Zeilenabstand
\setcounter{secnumdepth}{-1} % keine Nummerierung der Überschriften
%
%1
%%%%%%%%%%%%%%%%%%%%%%%%%%%%%%%%%%%%%%%%%%%%%%%%%%%%%%%%%%%%%%%%%
%%%%%%%%%%%%%%%%%%%%%% DOKUMENT - ANFANG %%%%%%%%%%%%%%%%%%%%%%%%
%%%%%%%%%%%%%%%%%%%%%%%%%%%%%%%%%%%%%%%%%%%%%%%%%%%%%%%%%%%%%%%%%
%
%

\begin{document}
\multiplechoice[5]{  %Anzahl der Antwortmoeglichkeiten, Standard: 5
				L1={test 1t},   %1. Antwortmoeglichkeit 
				L2={test 2},   %2. Antwortmoeglichkeit
				L3={test 3},   %3. Antwortmoeglichkeit
				L4={test 4},   %4. Antwortmoeglichkeit
				L5={test 5},	 %5. Antwortmoeglichkeit
				L6={},	 %6. Antwortmoeglichkeit
				L7={},	 %7. Antwortmoeglichkeit
				L8={},	 %8. Antwortmoeglichkeit
				L9={},	 %9. Antwortmoeglichkeit
				%% LOESUNG: %%
				A1=1,  % 1. Antwort
				A2=0,	 % 2. Antwort
				A3=0,  % 3. Antwort
				A4=0,  % 4. Antwort
				A5=0,  % 5. Antwort
				}


\tfmultiplechoice[5]{Statement}{  %Anzahl der Antwortmoeglichkeiten, Standard: 5
						L1={das},   %1. Antwortmoeglichkeit 
						L2={ist},   %2. Antwortmoeglichkeit
						L3={ein},   %3. Antwortmoeglichkeit
						L4={test},   %4. Antwortmoeglichkeit
						L5={wirklich},	 %5. Antwortmoeglichkeit
						L6={},	 %6. Antwortmoeglichkeit
						L7={},	 %7. Antwortmoeglichkeit
						L8={},	 %8. Antwortmoeglichkeit
						L9={},	 %9. Antwortmoeglichkeit
						%% LOESUNG: %%
						A1=0,  % 1. Antwort
						A2=0,	 % 2. Antwort
						A3=0,  % 3. Antwort
						A4=0,  % 4. Antwort
						A5=0,  % 5. Antwort
						}
\end{document}