

\documentclass[a4paper,12pt]{article}
\usepackage{geometry}
\geometry{a4paper,left=18mm,right=18mm, top=2cm, bottom=2cm}
 

\usepackage{lmodern}
\usepackage[T1]{fontenc}
\usepackage[utf8]{inputenc}
\usepackage[ngerman]{babel}
\usepackage[solution_on, random=0, info_off]{srdp-mathematik} % solution_on/off, random, info_on/off
%
\usepackage{blindtext}
%
%\pagestyle{plain} %PAGESTYLE: empty, plain
%\onehalfspacing %Zeilenabstand
%\setcounter{secnumdepth}{-1} % keine Nummerierung der Überschriften
%
%
%%%%%%%%%%%%%%%%%%%%%%%%%%%%%%%%%%%%%%%%%%%%%%%%%%%%%%%%%%%%%%%%%
%%%%%%%%%%%%%%%%%%%%%% DOKUMENT - ANFANG %%%%%%%%%%%%%%%%%%%%%%%%
%%%%%%%%%%%%%%%%%%%%%%%%%%%%%%%%%%%%%%%%%%%%%%%%%%%%%%%%%%%%%%%%%
%
%

%\NewEnviron{test}{
%	\begin{enumerate}
%		\item {\BODY}
%	\end{enumerate}
%}

\begin{document}
%\begin{test}
%Written text
%
%\begin{center}
%dasd
%\end{center}
%
%\end{test}

\begin{beispiel}{3} %PUNKTE DES BEISPIELS
Kreuze an.

\multiplechoice[5]{  %Anzahl der Antwortmoeglichkeiten, Standard: 5
				L1={daddad},   %1. Antwortmoeglichkeit 
				L2={dasd},   %2. Antwortmoeglichkeit
				L3={dasd},   %3. Antwortmoeglichkeit
				L4={dsdaa},   %4. Antwortmoeglichkeit
				L5={dasq},	 %5. Antwortmoeglichkeit
				L6={},	 %6. Antwortmoeglichkeit
				L7={},	 %7. Antwortmoeglichkeit
				L8={},	 %8. Antwortmoeglichkeit
				L9={},	 %9. Antwortmoeglichkeit
				%% LOESUNG: %%
				A1=1,  % 1. Antwort
				A2=3,	 % 2. Antwort
				A3=5,  % 3. Antwort
				A4=0,  % 4. Antwort
				A5=0,  % 5. Antwort
				}
\end{beispiel}

\begin{beispiel}{3} %PUNKTE DES BEISPIELS
Kreuze an.

\multiplechoice[5]{  %Anzahl der Antwortmoeglichkeiten, Standard: 5
				L1={daddad},   %1. Antwortmoeglichkeit 
				L2={dasd},   %2. Antwortmoeglichkeit
				L3={dasd},   %3. Antwortmoeglichkeit
				L4={dsdaa},   %4. Antwortmoeglichkeit
				L5={dasq},	 %5. Antwortmoeglichkeit
				L6={},	 %6. Antwortmoeglichkeit
				L7={},	 %7. Antwortmoeglichkeit
				L8={},	 %8. Antwortmoeglichkeit
				L9={},	 %9. Antwortmoeglichkeit
				%% LOESUNG: %%
				A1=1,  % 1. Antwort
				A2=3,	 % 2. Antwort
				A3=5,  % 3. Antwort
				A4=0,  % 4. Antwort
				A5=0,  % 5. Antwort
				}
\end{beispiel}

\begin{beispiel}{3} %PUNKTE DES BEISPIELS
Kreuze an.

\multiplechoice[5]{  %Anzahl der Antwortmoeglichkeiten, Standard: 5
				L1={daddad},   %1. Antwortmoeglichkeit 
				L2={dasd},   %2. Antwortmoeglichkeit
				L3={dasd},   %3. Antwortmoeglichkeit
				L4={dsdaa},   %4. Antwortmoeglichkeit
				L5={dasq},	 %5. Antwortmoeglichkeit
				L6={},	 %6. Antwortmoeglichkeit
				L7={},	 %7. Antwortmoeglichkeit
				L8={},	 %8. Antwortmoeglichkeit
				L9={},	 %9. Antwortmoeglichkeit
				%% LOESUNG: %%
				A1=1,  % 1. Antwort
				A2=3,	 % 2. Antwort
				A3=5,  % 3. Antwort
				A4=0,  % 4. Antwort
				A5=0,  % 5. Antwort
				}
\end{beispiel}

\notenschluessel[1/2]{0.91}{0.8}{0.64}{0.5}
\end{document}


% Aufgabenliste: 4, 10